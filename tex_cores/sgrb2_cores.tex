\newcommand{\ncores}{138\xspace}
\begin{document}

\sections{Observation and Data Reduction}
Data were acquired as part of ALMA project 2013.1.00269.S.  Observations were
taken with the 12m Total Power array, the ALMA 7m array, and in two
configurations with the ALMA 12m array.  The setup included the maximum allowed
number of channels, 15360, across 4 spectral windows in a single polarization;
the single-polarization mode was adopted to support moderate spectral resolution
across the broad bandwidth.

The ALMA QA2 calibrated measurement sets were combined to make a single
high-resolution, high-dynamic range data set.  We imaged the continuum jointly
across all four bands, and found that the central regions surrounding Sgr B2M
were severely affected by artifacts that could not be cleaned out.  We
therefore ran 3 iterations of phase-only self-calibration and one iteration of
amplitude + phase self-calibration to yield a substantially improved image.
The total dynamic range, measured as the peak brightness in Sgr B2 to the RMS
noise in a signal-free region of the image, is 22000 (noise $\sim0.08$
mJy/beam), while the dynamic range within one primary beam ($\sim0.5$\arcmin)
of Sgr B2M is only 3700 (noise $\sim0.5$ mJy/beam).  Because of the dynamic
range limitations, and an empirical determination that clean did not converge
if allowed to go too deep, we cleaned to a threshold of 0.5 mJy/beam across the
image.

We also produced cubes of all of the spectral lines.  These were lightly cleaned
with only 200 iterations of cleaning.  No self-calibration was applied.

\section{Analysis}

\subsection{Continuum Source Identification}
We selected continuum point sources as candidate cores or protostars by eye.
An automated selection is not viable across the majority of the field because
there are many extended \hii regions that dominate the overall map emission.  A
future automated selection algorithm may work if images at comparable
resolution at other frequencies become availabe; the \hii-region sources could
then be excluded.  Additionally, however, there are substantial imaging
artifacts produced by the extremely bright emission sources in Sgr B2 M ($S_{3
mm,max} > 0.8$ Jy) and Sgr B2 N ($S_{3 mm,max} > 0.3$ Jy) that make automated
source identification particularly challenging in the regions they are most
common.

\section{Results}
We detected \ncores compact continuum sources.  The majority of these are
likely to be dust-dominated protostellar envelopes, though many are free-free
dominated hypercompact \hii regions.  They are unlikely to be dusty prestellar
cores, since they are predominantly unresolved or barely resolved, with
$R<1\arcsec$ ($R<8500$ AU).


Possible approaches:
-Create a column density map from one or more of the Herschel / Laboca / Bolocam / SCUBA maps,
then make some cumulative N(cores) vs column plots.  Assess SF thresholds.

\end{document}

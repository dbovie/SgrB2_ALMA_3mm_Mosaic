\documentclass[twocolumn]{aastex61}
%\documentclass[defaultstyle,11pt]{thesis}
%\documentclass[]{report}
%\documentclass[]{article}
%\usepackage{aastex_hack}
%\usepackage{deluxetable}
%\documentclass[preprint]{aastex}
%\documentclass{aa}

\newcommand{\titlerunning}[1]{\shorttitle{#1}}
\newcommand{\authorrunning}[1]{\shortauthors{#1}}

\newcommand*\inst[1]{\unskip\hbox{\@textsuperscript{\normalfont$#1$}}}

%\newcount\aa@nbinstitutes
%
%\newcounter{aa@institutecnt}

\newcommand*\institute[1]{
  \begingroup
    \let\and\relax
    \renewcommand*\inst[1]{}%
    \renewcommand*\thanks[1]{}%
    \renewcommand*\email[1]{}%
    %\let\@@protect\protect
    %\let\protect\@unexpandable@protect
    %\global\aa@nbinstitutes \z@
    %\expandafter\aa@cntinstitutes\aa@institute\and\aa@nil\and
    %\restore@protect
  \endgroup
  \newcommand{\institutions}{#1}
}%


%\renewcommand{\abstract}[1]{
%\begin{abstract}
%    #1
%\end{abstract}
%}

%\renewcommand\ion[2]{#1$\;${%
%\ifx\@currsize\normalsize\small \else
%\ifx\@currsize\small\footnotesize \else
%\ifx\@currsize\footnotesize\scriptsize \else
%\ifx\@currsize\scriptsize\tiny \else
%\ifx\@currsize\large\normalsize \else
%\ifx\@currsize\Large\large
%\fi\fi\fi\fi\fi\fi
%\rmfamily\@Roman{#2}}\relax}% 
%
%\renewcommand{ion}[2]{#1}{#2}

\renewcommand{\ion}[2]{\textup{#1\,\textsc{\lowercase{#2}}}}

%\newcommand{\uchii}{\ensuremath{\mathrm{\ion{UCH}{2}}}\xspace}
%\newcommand{\UCHII}{\ensuremath{\mathrm{\ion{UCH}{2}}}\xspace}
%\newcommand{\hchii}{\ensuremath{\mathrm{\ion{HCH}{2}}}\xspace}
%\newcommand{\HCHII}{\ensuremath{\mathrm{\ion{HCH}{2}}}\xspace}
%\newcommand{\hii}  {\ensuremath{\mathrm{\ion{H}{2}}}\xspace}

%\input{aamacros.tex}

\pdfminorversion=4


%%%%%%%%%%%%%%%%%%%%%%%%%%%%%%%%%%%%%%%%%%%%%%%%%%%%%%%%%%%%%%%%
%%%%%%%%%%%  see documentation for information about  %%%%%%%%%%
%%%%%%%%%%%  the options (11pt, defaultstyle, etc.)   %%%%%%%%%%
%%%%%%%  http://www.colorado.edu/its/docs/latex/thesis/  %%%%%%%
%%%%%%%%%%%%%%%%%%%%%%%%%%%%%%%%%%%%%%%%%%%%%%%%%%%%%%%%%%%%%%%%
%		\documentclass[typewriterstyle]{thesis}
% 		\documentclass[modernstyle]{thesis}
% 		\documentclass[modernstyle,11pt]{thesis}
%	 	\documentclass[modernstyle,12pt]{thesis}

%%%%%%%%%%%%%%%%%%%%%%%%%%%%%%%%%%%%%%%%%%%%%%%%%%%%%%%%%%%%%%%%
%%%%%%%%%%%    load any packages which are needed    %%%%%%%%%%%
%%%%%%%%%%%%%%%%%%%%%%%%%%%%%%%%%%%%%%%%%%%%%%%%%%%%%%%%%%%%%%%%
\usepackage{latexsym}		% to get LASY symbols
\usepackage{graphicx}		% to insert PostScript figures
%\usepackage{deluxetable}
\usepackage{rotating}		% for sideways tables/figures
\usepackage{natbib}  % Requires natbib.sty, available from http://ads.harvard.edu/pubs/bibtex/astronat/
\usepackage{savesym}
%\usepackage{pdflscape}
\usepackage{amssymb}
\usepackage{morefloats}
%\savesymbol{singlespace}
\savesymbol{doublespace}
%\usepackage{wrapfig}
%\usepackage{setspace}
\usepackage{xspace}
\usepackage{color}
%\usepackage{multicol}
\usepackage{mdframed}
\usepackage{url}
\usepackage{subfigure}
%\usepackage{emulateapj}
%\usepackage{lscape}
\usepackage{grffile}
\usepackage{standalone}
\standalonetrue
\usepackage{import}
\usepackage[utf8]{inputenc}
\usepackage{longtable}
\usepackage{booktabs}
\usepackage[yyyymmdd,hhmmss]{datetime}
\usepackage{fancyhdr}
\usepackage[colorlinks=true,citecolor=blue,linkcolor=cyan]{hyperref}
\usepackage{ifpdf}







\newcommand{\paa}{Pa\ensuremath{\alpha}}
\newcommand{\brg}{Br\ensuremath{\gamma}}
\newcommand{\msun}{\ensuremath{M_{\odot}}\xspace}			%  Msun
\newcommand{\mdot}{\ensuremath{\dot{M}}\xspace}
\newcommand{\lsun}{\ensuremath{L_{\odot}}\xspace}			%  Lsun
\newcommand{\rsun}{\ensuremath{R_{\odot}}\xspace}			%  Rsun
\newcommand{\lbol}{\ensuremath{L_{\mathrm{bol}}\xspace}}	%  Lbol
\newcommand{\ks}{K\ensuremath{_{\mathrm{s}}}}		%  Ks
\newcommand{\hh}{\ensuremath{\textrm{H}_{2}}\xspace}			%  H2
\newcommand{\dens}{\ensuremath{n(\hh) [\percc]}\xspace}
\newcommand{\formaldehyde}{\ensuremath{\textrm{H}_2\textrm{CO}}\xspace}
\newcommand{\formamide}{\ensuremath{\textrm{NH}_2\textrm{CHO}}\xspace}
\newcommand{\formaldehydeIso}{\ensuremath{\textrm{H}_2~^{13}\textrm{CO}}\xspace}
\newcommand{\methanol}{\ensuremath{\textrm{CH}_3\textrm{OH}}\xspace}
\newcommand{\ortho}{\ensuremath{\textrm{o-H}_2\textrm{CO}}\xspace}
\newcommand{\para}{\ensuremath{\textrm{p-H}_2\textrm{CO}}\xspace}
\newcommand{\oneone}{\ensuremath{1_{1,0}-1_{1,1}}\xspace}
\newcommand{\twotwo}{\ensuremath{2_{1,1}-2_{1,2}}\xspace}
\newcommand{\threethree}{\ensuremath{3_{1,2}-3_{1,3}}\xspace}
\newcommand{\threeohthree}{\ensuremath{3_{0,3}-2_{0,2}}\xspace}
\newcommand{\threetwotwo}{\ensuremath{3_{2,2}-2_{2,1}}\xspace}
\newcommand{\threetwoone}{\ensuremath{3_{2,1}-2_{2,0}}\xspace}
\newcommand{\fourtwotwo}{\ensuremath{4_{2,2}-3_{1,2}}\xspace} % CH3OH 218.4 GHz
\newcommand{\methylcyanide}{\ensuremath{\textrm{CH}_{3}\textrm{CN}}\xspace}
\newcommand{\ketene}{\ensuremath{\textrm{H}_{2}\textrm{CCO}}\xspace}
\newcommand{\ethylcyanide}{\ensuremath{\textrm{CH}_3\textrm{CH}_2\textrm{CN}}\xspace}
\newcommand{\cyanoacetylene}{\ensuremath{\textrm{HC}_{3}\textrm{N}}\xspace}
\newcommand{\methylformate}{\ensuremath{\textrm{CH}_{3}\textrm{OCHO}}\xspace}
\newcommand{\dimethylether}{\ensuremath{\textrm{CH}_{3}\textrm{OCH}_{3}}\xspace}
\newcommand{\gaucheethanol}{\ensuremath{\textrm{g-CH}_3\textrm{CH}_2\textrm{OH}}\xspace}
\newcommand{\acetone}{\ensuremath{\left[\textrm{CH}_{3}\right]_2\textrm{CO}}\xspace}
\newcommand{\methyleneamidogen}{\ensuremath{\textrm{H}_{2}\textrm{CN}}\xspace}
\newcommand{\Rone}{\ensuremath{\para~S_{\nu}(\threetwoone) / S_{\nu}(\threeohthree)}\xspace}
\newcommand{\Rtwo}{\ensuremath{\para~S_{\nu}(\threetwotwo) / S_{\nu}(\threetwoone)}\xspace}
\newcommand{\JKaKc}{\ensuremath{J_{K_a K_c}}}
\newcommand{\water}{H$_{2}$O\xspace}		%  H2O
\newcommand{\feii}{\ion{Fe}{ii}\xspace}		%  FeII

\newcommand{\uchii}{\ion{UCH}{ii}\xspace}
\newcommand{\UCHII}{\ion{UCH}{ii}\xspace}
\newcommand{\hchii}{\ion{HCH}{ii}\xspace}
\newcommand{\HCHII}{\ion{HCH}{ii}\xspace}
\newcommand{\hii}{\ion{H}{ii}\xspace}

\newcommand{\hi}{H~{\sc i}\xspace}
\newcommand{\Hii}{\hii}
\newcommand{\HII}{\hii}
\newcommand{\Xform}{\ensuremath{X_{\formaldehyde}}}
\newcommand{\kms}{\textrm{km~s}\ensuremath{^{-1}}\xspace}	%  km s-1
\newcommand{\nsample}{456\xspace}
\newcommand{\CFR}{5\xspace} % nMPC / 0.25 / 2 (6 for W51 once, 8 for W51 twice) REFEDIT: With f_observed=0.3, becomes 3/2./0.3 = 5
\newcommand{\permyr}{\ensuremath{\mathrm{Myr}^{-1}}\xspace}
\newcommand{\pers}{\ensuremath{\mathrm{s}^{-1}}\xspace}
\newcommand{\tsuplim}{0.5\xspace} % upper limit on starless timescale
\newcommand{\ncandidates}{18\xspace}
\newcommand{\mindist}{8.7\xspace}
\newcommand{\rcluster}{2.5\xspace}
\newcommand{\ncomplete}{13\xspace}
\newcommand{\middistcut}{13.0\xspace}
\newcommand{\nMPC}{3\xspace} % only count W51 once.  W51, W49, G010
\newcommand{\obsfrac}{30}
\newcommand{\nMPCtot}{10\xspace} % = nmpc / obsfrac
\newcommand{\nMPCtoterr}{6\xspace} % = sqrt(nmpc) / obsfrac
\newcommand{\plaw}{2.1\xspace}
\newcommand{\plawerr}{0.3\xspace}
\newcommand{\mmin}{\ensuremath{10^4~\msun}\xspace}
%\newcommand{\perkmspc}{\textrm{per~km~s}\ensuremath{^{-1}}\textrm{pc}\ensuremath{^{-1}}\xspace}	%  km s-1 pc-1
\newcommand{\kmspc}{\textrm{km~s}\ensuremath{^{-1}}\textrm{pc}\ensuremath{^{-1}}\xspace}	%  km s-1 pc-1
\newcommand{\sqcm}{cm$^{2}$\xspace}		%  cm^2
\newcommand{\percc}{\ensuremath{\textrm{cm}^{-3}}\xspace}
\newcommand{\perpc}{\ensuremath{\textrm{pc}^{-1}}\xspace}
\newcommand{\persc}{\ensuremath{\textrm{cm}^{-2}}\xspace}
\newcommand{\persr}{\ensuremath{\textrm{sr}^{-1}}\xspace}
\newcommand{\peryr}{\ensuremath{\textrm{yr}^{-1}}\xspace}
\newcommand{\perkmspc}{\textrm{km~s}\ensuremath{^{-1}}\textrm{pc}\ensuremath{^{-1}}\xspace}	%  km s-1 pc-1
\newcommand{\perkms}{\textrm{per~km~s}\ensuremath{^{-1}}\xspace}	%  km s-1 
\newcommand{\um}{\ensuremath{\mu \textrm{m}}\xspace}    % micron
\newcommand{\microjy}{\ensuremath{\mu\textrm{Jy}}\xspace}    % micron
\newcommand{\mum}{\um}
\newcommand{\htwo}{\ensuremath{\textrm{H}_2}}
\newcommand{\Htwo}{\ensuremath{\textrm{H}_2}}
\newcommand{\HtwoO}{\ensuremath{\textrm{H}_2\textrm{O}}}
\newcommand{\htwoo}{\ensuremath{\textrm{H}_2\textrm{O}}}
\newcommand{\ha}{\ensuremath{\textrm{H}\alpha}}
\newcommand{\hb}{\ensuremath{\textrm{H}\beta}}
\newcommand{\so}{SO~\ensuremath{5_6-4_5}\xspace}
\newcommand{\SO}{SO~\ensuremath{1_2-1_1}\xspace}
\newcommand{\ammonia}{NH\ensuremath{_3}\xspace}
\newcommand{\twelveco}{\ensuremath{^{12}\textrm{CO}}\xspace}
\newcommand{\thirteenco}{\ensuremath{^{13}\textrm{CO}}\xspace}
\newcommand{\ceighteeno}{\ensuremath{\textrm{C}^{18}\textrm{O}}\xspace}
\def\ee#1{\ensuremath{\times10^{#1}}}
\newcommand{\degrees}{\ensuremath{^{\circ}}}
% can't have \degree because I'm getting a degree...
\newcommand{\lowirac}{800}
\newcommand{\highirac}{8000}
\newcommand{\lowmips}{600}
\newcommand{\highmips}{5000}
\newcommand{\perbeam}{\ensuremath{\textrm{beam}^{-1}}}
\newcommand{\ds}{\ensuremath{\textrm{d}s}}
\newcommand{\dnu}{\ensuremath{\textrm{d}\nu}}
\newcommand{\dv}{\ensuremath{\textrm{d}v}}
\def\secref#1{Section \ref{#1}}
\def\eqref#1{Equation \ref{#1}}
\def\facility#1{#1}
%\newcommand{\arcmin}{'}

\newcommand{\necluster}{Sh~2-233IR~NE}
\newcommand{\swcluster}{Sh~2-233IR~SW}
\newcommand{\region}{IRAS 05358}

\newcommand{\nwfive}{40}
\newcommand{\nouter}{15}

\newcommand{\vone}{{\rm v}1.0\xspace}
\newcommand{\vtwo}{{\rm v}2.0\xspace}
\newcommand\mjysr{\ensuremath{{\rm MJy~sr}^{-1}}}
\newcommand\jybm{\ensuremath{{\rm Jy~bm}^{-1}}}
\newcommand\nbolocat{8552\xspace}
\newcommand\nbolocatnew{548\xspace}
\newcommand\nbolocatnonew{8004\xspace} % = nbolocat-nbolocatnew
%\renewcommand\arcdeg{\mbox{$^\circ$}\xspace} 
%\renewcommand\arcmin{\mbox{$^\prime$}\xspace} 
%\renewcommand\arcsec{\mbox{$^{\prime\prime}$}\xspace} 

\newcommand{\todo}[1]{\textcolor{red}{#1}}
\newcommand{\okinfinal}[1]{{#1}}
%% only needed if not aastex
%\newcommand{\keywords}[1]{}
%\newcommand{\email}[1]{}
%\newcommand{\affil}[1]{}


%aastex hack
%\newcommand\arcdeg{\mbox{$^\circ$}}%
%\newcommand\arcmin{\mbox{$^\prime$}\xspace}%
%\newcommand\arcsec{\mbox{$^{\prime\prime}$}\xspace}%

%\newcommand\epsscale[1]{\gdef\eps@scaling{#1}}
%
%\newcommand\plotone[1]{%
% \typeout{Plotone included the file #1}
% \centering
% \leavevmode
% \includegraphics[width={\eps@scaling\columnwidth}]{#1}%
%}%
%\newcommand\plottwo[2]{{%
% \typeout{Plottwo included the files #1 #2}
% \centering
% \leavevmode
% \columnwidth=.45\columnwidth
% \includegraphics[width={\eps@scaling\columnwidth}]{#1}%
% \hfil
% \includegraphics[width={\eps@scaling\columnwidth}]{#2}%
%}}%


%\newcommand\farcm{\mbox{$.\mkern-4mu^\prime$}}%
%\let\farcm\farcm
%\newcommand\farcs{\mbox{$.\!\!^{\prime\prime}$}}%
%\let\farcs\farcs
%\newcommand\fp{\mbox{$.\!\!^{\scriptscriptstyle\mathrm p}$}}%
%\newcommand\micron{\mbox{$\mu$m}}%
%\def\farcm{%
% \mbox{.\kern -0.7ex\raisebox{.9ex}{\scriptsize$\prime$}}%
%}%
%\def\farcs{%
% \mbox{%
%  \kern  0.13ex.%
%  \kern -0.95ex\raisebox{.9ex}{\scriptsize$\prime\prime$}%
%  \kern -0.1ex%
% }%
%}%

\def\Figure#1#2#3#4#5{
\begin{figure*}[!htp]
\includegraphics[scale=#4,width=#5]{#1}
\caption{#2}
\label{#3}
\end{figure*}
}

\def\WrapFigure#1#2#3#4#5#6{
\begin{wrapfigure}{#6}{0.5\textwidth}
\includegraphics[scale=#4,width=#5]{#1}
\caption{#2}
\label{#3}
\end{wrapfigure}
}

% % #1 - filename
% % #2 - caption
% % #3 - label
% % #4 - epsscale
% % #5 - R or L?
% \def\WrapFigure#1#2#3#4#5#6{
% \begin{wrapfigure}[#6]{#5}{0.45\textwidth}
% %  \centercaption
% %  \vspace{-14pt}
%   \epsscale{#4}
%   \includegraphics[scale=#4]{#1}
%   \caption{#2}
%   \label{#3}
% \end{wrapfigure}
% }

\def\RotFigure#1#2#3#4#5{
\begin{sidewaysfigure*}[!htp]
\includegraphics[scale=#4,width=#5]{#1}
\caption{#2}
\label{#3}
\end{sidewaysfigure*}
}

\def\FigureSVG#1#2#3#4{
\begin{figure*}[!htp]
    \def\svgwidth{#4}
    \input{#1}
    \caption{#2}
    \label{#3}
\end{figure*}
}

% originally intended to be included in a two-column paper
% this is in includegraphics: ,width=3in
% but, not for thesis
\def\OneColFigure#1#2#3#4#5{
\begin{figure}[!htpb]
\epsscale{#4}
\includegraphics[scale=#4,angle=#5]{#1}
\caption{#2}
\label{#3}
\end{figure}
}

\def\SubFigure#1#2#3#4#5{
\begin{figure*}[!htp]
\addtocounter{figure}{-1}
\epsscale{#4}
\includegraphics[angle=#5]{#1}
\caption{#2}
\label{#3}
\end{figure*}
}

%\def\FigureTwo#1#2#3#4#5{
%\begin{figure*}[!htp]
%\epsscale{#5}
%\plottwo{#1}{#2}
%\caption{#3}
%\label{#4}
%\end{figure*}
%}

\def\FigureTwo#1#2#3#4#5#6{
\begin{figure*}[!htp]
\subfigure[]{ \includegraphics[scale=#5,width=#6]{#1} }
\subfigure[]{ \includegraphics[scale=#5,width=#6]{#2} }
\caption{#3}
\label{#4}
\end{figure*}
}

\def\FigureTwoAA#1#2#3#4#5#6{
\begin{figure*}[!htp]
\subfigure[]{ \includegraphics[scale=#5,width=#6]{#1} }
\subfigure[]{ \includegraphics[scale=#5,width=#6]{#2} }
\caption{#3}
\label{#4}
\end{figure*}
}

\newenvironment{rotatepage}%
{}{}
   %{\pagebreak[4]\afterpage\global\pdfpageattr\expandafter{\the\pdfpageattr/Rotate 90}}%
   %{\pagebreak[4]\afterpage\global\pdfpageattr\expandafter{\the\pdfpageattr/Rotate 0}}%


\def\RotFigureTwoAA#1#2#3#4#5#6{
\begin{rotatepage}
\begin{sidewaysfigure*}[!htp]
\subfigure[]{ \includegraphics[scale=#5,width=#6]{#1} }
\\
\subfigure[]{ \includegraphics[scale=#5,width=#6]{#2} }
\caption{#3}
\label{#4}
\end{sidewaysfigure*}
\end{rotatepage}
}

\def\RotFigureThreeAA#1#2#3#4#5#6#7{
\begin{rotatepage}
\begin{sidewaysfigure*}[!htp]
\subfigure[]{ \includegraphics[scale=#6,width=#7]{#1} }
\\
\subfigure[]{ \includegraphics[scale=#6,width=#7]{#2} }
\\
\subfigure[]{ \includegraphics[scale=#6,width=#7]{#3} }
\caption{#4}
\label{#5}
\end{sidewaysfigure*}
\end{rotatepage}
\clearpage
}

\def\FigureThreeAA#1#2#3#4#5#6#7{
\begin{figure*}[!htp]
\subfigure[]{ \includegraphics[scale=#6,width=#7]{#1} }
\subfigure[]{ \includegraphics[scale=#6,width=#7]{#2} }
\subfigure[]{ \includegraphics[scale=#6,width=#7]{#3} }
\caption{#4}
\label{#5}
\end{figure*}
}



\def\SubFigureTwo#1#2#3#4#5{
\begin{figure*}[!htp]
\addtocounter{figure}{-1}
\epsscale{#5}
\plottwo{#1}{#2}
\caption{#3}
\label{#4}
\end{figure*}
}

\def\FigureFour#1#2#3#4#5#6{
\begin{figure*}[!htp]
\subfigure[]{ \includegraphics[width=3in]{#1} }
\subfigure[]{ \includegraphics[width=3in]{#2} }
\subfigure[]{ \includegraphics[width=3in]{#3} }
\subfigure[]{ \includegraphics[width=3in]{#4} }
\caption{#5}
\label{#6}
\end{figure*}
}

\def\FigureFourPDF#1#2#3#4#5#6{
\begin{figure*}[!htp]
\subfigure[]{ \includegraphics[width=3in,type=pdf,ext=.pdf,read=.pdf]{#1} }
\subfigure[]{ \includegraphics[width=3in,type=pdf,ext=.pdf,read=.pdf]{#2} }
\subfigure[]{ \includegraphics[width=3in,type=pdf,ext=.pdf,read=.pdf]{#3} }
\subfigure[]{ \includegraphics[width=3in,type=pdf,ext=.pdf,read=.pdf]{#4} }
\caption{#5}
\label{#6}
\end{figure*}
}

\def\FigureThreePDF#1#2#3#4#5{
\begin{figure*}[!htp]
\subfigure[]{ \includegraphics[width=3in,type=pdf,ext=.pdf,read=.pdf]{#1} }
\subfigure[]{ \includegraphics[width=3in,type=pdf,ext=.pdf,read=.pdf]{#2} }
\subfigure[]{ \includegraphics[width=3in,type=pdf,ext=.pdf,read=.pdf]{#3} }
\caption{#4}
\label{#5}
\end{figure*}
}

\def\Table#1#2#3#4#5{
%\renewcommand{\thefootnote}{\alph{footnote}}
\begin{table}
\caption{#2}
\label{#3}
    \begin{tabular}{#1}
        \hline\hline
        #4
        \hline
        #5
        \hline
    \end{tabular}
\end{table}
%\renewcommand{\thefootnote}{\arabic{footnote}}
}


%\def\Table#1#2#3#4#5#6{
%%\renewcommand{\thefootnote}{\alph{footnote}}
%\begin{deluxetable}{#1}
%\tablewidth{0pt}
%\tabletypesize{\footnotesize}
%\tablecaption{#2}
%\tablehead{#3}
%\startdata
%\label{#4}
%#5
%\enddata
%\bigskip
%#6
%\end{deluxetable}
%%\renewcommand{\thefootnote}{\arabic{footnote}}
%}

%\def\tablenotetext#1#2{
%\footnotetext[#1]{#2}
%}

% \def\LongTable#1#2#3#4#5#6#7#8{
% % required to get tablenotemark to work: http://www2.astro.psu.edu/users/stark/research/psuthesis/longtable.html
% \renewcommand{\thefootnote}{\alph{footnote}}
% \begin{longtable}{#1}
% \caption[#2]{#2}
% \label{#4} \\
% 
%  \\
% \hline 
% #3 \\
% \hline
% \endfirsthead
% 
% \hline
% #3 \\
% \hline
% \endhead
% 
% \hline
% \multicolumn{#8}{r}{{Continued on next page}} \\
% \hline
% \endfoot
% 
% \hline 
% \endlastfoot
% #7 \\
% 
% #5
% \hline
% #6 \\
% 
% \end{longtable}
% \renewcommand{\thefootnote}{\arabic{footnote}}
% }

\def\TallFigureTwo#1#2#3#4#5#6{
\begin{figure*}[htp]
\epsscale{#5}
\subfigure[]{ \includegraphics[width=#6]{#1} }
\subfigure[]{ \includegraphics[width=#6]{#2} }
\caption{#3}
\label{#4}
\end{figure*}
}

		% file containing author's macro definitions

\input{ncores}
\newcommand{\dsgrb}{8.5 kpc\xspace}
\newcommand{\percent}{\%\xspace}

\begin{document}
\title{A catalog of 3mm point sources in the Sgr B2 cloud: signs of extended star formation in a CMZ cloud}
\titlerunning{Sgr B2 ALMA}
\authorrunning{Ginsburg et al}
% for future reference, this is probably a better approach:
% https://github.com/dfm/peerless/blob/af483ced97045c213650ed807c430b2f87d2c8c9/document/ms.tex#L104
% assuming it's compatible with A&A
%\newcommand{\nrao}{$^{1}$}
%\newcommand{\eso}{$^{2}$}
\newcommand{\nraojansky}{\affiliation{\it{Jansky fellow of the National Radio Astronomy Observatory, 1003 Lopezville Rd, Socorro, NM 87801 USA }}}
\newcommand{\nrao}{\affiliation{\it{National Radio Astronomy Observatory, 1003 Lopezville Rd, Socorro, NM 87801 USA }}}
\newcommand{\nraocv}{\affiliation{\it{National Radio Astronomy Observatory, 520 Edgemont Rd, Charlottesville, VA 22903, USA }}}
\newcommand{\eso}{ \affiliation{\it{ European Southern Observatory, Karl-Schwarzschild-Stra{\ss}e 2, D-85748 Garching bei München, Germany } } }


\newcommand{\radboud}{\affiliation{\it{Department of Astrophysics/IMAPP, Radboud University Nijmegen, PO Box 9010, 6500 GL Nijmegen, the Netherlands}}}
\newcommand{\allegro}{\affiliation{\it{ALLEGRO/Leiden Observatory, Leiden University, PO Box 9513, 2300 RA Leiden, the Netherlands}}}
\newcommand{\zah}{\affiliation{\it{Astronomisches Rechen-Institut, Zentrum f{\"u}r Astronomie der Universit{\"a}t Heidelberg, M{\"o}nchhofstra{\ss}e 12-14, 69120 Heidelberg, Germany}}}
\newcommand{\casa}{\affiliation{\it{CASA, University of Colorado, 389-UCB, Boulder, CO 80309}} }
\newcommand{\jodrell}{\affiliation{\it{Jodrell Bank Centre for Astrophysics, School of Physics and Astronomy, University of Manchester, Oxford Road, Manchester M13 9PL, UK}}}
\newcommand{\morelia}{\affiliation{\it{Instituto de Radioastronom{\'i}a y Astrof{\'i}sica, UNAM, A.P. 3-72, Xangari, Morelia, 58089, Mexico}}}
\newcommand{\sjsu}{\affiliation{\it{{San Jose State University, One Washington Square, San Jose, CA 95192}}}}
\newcommand{\herts}{\affiliation{\it{Centre for Astrophysics Research, University of Hertfordshire, College Lane, Hatfield, AL10 9AB, UK}}}
\newcommand{\uofa}{\affiliation{\it{Dept. of Physics, University of Alberta, Edmonton, Alberta, Canada}}}
\newcommand{\arcetri}{\affiliation{\it{INAF-Osservatorio Astrofisico di Arcetri, Largo E. Fermi 5, I-50125, Florence, Italy } } }
\newcommand{\exclus}{\affiliation{\it{Excellence Cluster Universe, Boltzman str. 2, D-85748 Garching bei M\"unchen, Germany } }}
\newcommand{\ljmu}{\affiliation{\it{Astrophysics Research Institute, Liverpool John Moores University, 146 Brownlow Hill, Liverpool L3 5RF, UK }}}
\newcommand{\koeln}{\affiliation{\it{I. Physikalisches Institut, Universi\"at zu K\"oln, Z\"ulpicher Str.\ 77, 50937 K\"oln, Germany}}}
\newcommand{\mpia}{\affiliation{\it{Max-Planck-Institute for Astronomy, Koenigstuhl 17, 69117 Heidelberg, Germany}}}
\newcommand{\agnesscott}{\affiliation{\it{Agnes Scott College, 141 E. College Ave., Decatur, GA 30030}}}
\newcommand{\chile}{\affiliation{\it{Departamento de Astronom{\'i}a, Universidad de Chile, Casilla 36-D, Santiago, Chile}}}
\newcommand{\leiden}{\affiliation{\it{Leiden Observatory, Leiden University, PO Box 9513, NL-2300 RA Leiden, the Netherlands }}}
\newcommand{\mpe}{\affiliation{\it{Max-Planck-Institut für extraterrestrische Physik, D-85748 Garching, Germany}}}
\newcommand{\boston}{\affiliation{\it{Boston University Astronomy Department, 725 Commonwealth Avenue, Boston, MA 02215, USA}}}
\newcommand{\cfa}{\affiliation{\it{Harvard-Smithsonian Center for Astrophysics, 60 Garden St. Cambridge, MA 02138}}}
\newcommand{\usf}{\affiliation{\it{University of South Florida, Physics Department, 4202 East Fowler Ave, ISA 2019 Tampa, FL 33620}}}
\newcommand{\uconn}{\affiliation{\it{University of Connecticut, Department of Physics, 2152 Hillside Rd., Storrs, CT 06269}}}

\newcommand{\jao}{\affiliation{\it{Joint ALMA Observatory, Alonso de Córdova 3107, Vitacura, Santiago, Chile}}}
\newcommand{\naoj}{\affiliation{\it{National Astronomical Observatory of Japan, Alonso de Córdova 3788, 61B Vitacura, Santiago, Chile}}}
\newcommand{\naojtwo}{\affiliation{\it{National Astronomical Observatory of Japan, 2-21-1 Osawa, Mitaka,Tokyo, 181-8588, Japan}}}

\author[0000-0001-6431-9633]{Adam Ginsburg}
\nraojansky
\eso

%\author{
%Adam Ginsburg{\nrao},
%\begin{flushleft}
%\institutions
%\end{flushleft}
%        }
%
%\institute{
%    {\nrao}{\it{National Radio Astronomy Observatory, Socorro, NM 87801 USA\\
%                      \email{aginsbur@nrao.edu} 
%                      }} \\
%    }
%
\correspondingauthor{Adam Ginsburg}
\email{aginsbur@nrao.edu; adam.g.ginsburg@gmail.com}


\author{John Bally}
\casa

\author{Ashley Barnes}
\ljmu

\author{Nate Bastian }
\ljmu

\author{Cara Battersby}
\cfa
\uconn

\author{Henrik Beuther}
\mpia

\author{Crystal Brogan}
\nraocv

\author{Yanett Contreras}
\leiden


\author{Joanna Corby}
\nraocv
\usf

\author{Jeremy Darling}
\casa

\author{Chris De~Pree}
\agnesscott

\author{Roberto Galv{\'a}n-Madrid}
\morelia


\author{Guido Garay}
\chile

\author{Jonathan Henshaw}
\mpia


\author{Todd Hunter}
\nraocv

\author{J.~M.~Diederik Kruijssen}
\zah

\author{Steven Longmore}
\ljmu

\author{Xing Lu}
\naojtwo

\author{Fanyi Meng}
\koeln

\author{Elisabeth A.C. Mills }
\sjsu
\boston

\author{Juergen Ott}
\nrao

\author{Jaime E. Pineda}
\mpe



\author{{\'A}lvaro S{\'a}nchez-Monge}
\koeln
\author{Peter Schilke}
\koeln
\author{Anika Schmiedeke}
\koeln
\mpe





\author{Daniel Walker}
\ljmu
\jao
\naoj

\author{David Wilner}
\cfa

\begin{abstract}
We report ALMA observations at 3 mm  of the extended Sgr B2 cloud. We detected
\ncores compact sources, the majority of which have extents 
smaller than 5000 AU.  These sources are predominantly protostars or
centrally condensed prestellar cores.  The spatial distribution of these sources
demonstrates that Sgr B2 is experiencing a highly extended star formation
event, not just an isolated `starburst' within the protocluster regions M, N,
and S.  While all protostars reside in regions of high column density, not all
regions of high column density possess a high density of protostars.  
These observations constitute the largest single, possibly coeval, sample of
forming massive stars.
% the high-density, low-SFR regions raise questions...
\end{abstract}

%\ifpdf
%\maketitle
%\fi

\section{Introduction}
The Central Molecular Zone (CMZ) of our Galaxy appears to be overall deficient
in star formation relative to the gas mass it contains \citep{Longmore2013a,
Kauffmann2016a,Kauffmann2016b,Barnes2016c,Barnes2017b}.  This deficiency
suggests that star formation laws, i.e., the empirical relations between
the star formation rate and gas density, are not universal.  The gas
conditions in the Galactic center provide a powerful lever-arm in a few
parameters \citep[e.g., pressure, temperature, velocity
dispersion][]{Ginsburg2016a,Immer2016a,Shetty2012a,Henshaw2016a} to assess the
influence of environmental effects on star formation.

The observations that have demonstrated the star formation deficiency compare
bulk tracers of star formation to $\gtrsim0.1$ pc resolution gas observations
\citep[e.g.]{Barnes2017b}.  More recently, high-resolution observations
of selected clouds in the CMZ have revealed very few star-forming cores
even when examined at high resolution and sensitivity
\citep{Rathborne2015a,Kauffmann2016a,Kauffmann2016b}.  The only sites with
obvious signs of ongoing star formation along the CMZ dust ridge are
the Sgr B2 N, M, and S protoclusters \citep{Schmiedeke2016a} and, at a much
lower level, Clouds C, D, and E \citep[][Walker et al, in prep; Barnes et al,
in prep]{Ginsburg2015b}.  These regions contain a handful of high-mass cores
detected with ALMA, but only a small number of protostars.

We report the first observations of extended, ongoing star formation in a
Galactic center cloud \emph{not} isolated to a centralized protocluster dust
clump.  We observed a $\sim15\times15$ pc section of the Sgr B2 cloud and
identified star formation along the entire molecular dust ridge known as Sgr B2
Deep South (DS).  These observations allow us to perform the first
star-counting based determination of the star formation rate within the
molecular gas of the CMZ.

We describe the observations in Section \ref{sec:observations}. In this paper,
we focus on the continuum sources, which we identify in Section
\ref{sec:contsources}.  We classify the sources in Section
\ref{sec:classification}.  

\FigureTwo
{figures/continuum_peak_MandN.pdf}
{figures/cores_on_continuum_peak_MandN.pdf}
{Images of the 3 mm continuum in the Sgr B2 M and N region.  The right figure
includes a red dot at the position of each identified continnuum pointlike
source.  The massive protocluster Sgr B2 M is the collection of \hii regions
and compact source in the lower half of the image.  The other massive
protocluster, Sgr B2 N, is in the center.  Imaging artifacts are evident at the
1-2 mJy bm$^{-1}$ level.
}
{fig:continuumMandN}{1}{0.5\textwidth}

\section{Observation and Data Reduction}
\label{sec:observations}
Data were acquired as part of ALMA project 2013.1.00269.S.  Observations were
taken with the 12m Total Power array, the ALMA 7m array, and in two
configurations with the ALMA 12m array.  The setup included the maximum allowed
number of channels, 30720, across 4 spectral windows in a single polarization;
the single-polarization mode was adopted to support moderate spectral resolution
across the broad bandwidth.

% analysis/measure_dynamic_range.py
The ALMA QA2 calibrated measurement sets were combined to make a single
high-resolution, high-dynamic range data set.  We imaged the continuum jointly
across all four bands, and found that the central regions surrounding Sgr B2 M
were severely affected by artifacts that could not be cleaned out.  We
therefore ran 3 iterations of phase-only self-calibration and two iterations of
amplitude + phase self-calibration, the latter using multi-scale
multi-frequency synthesis with two Taylor terms, to yield a substantially
improved image (see Appendix \ref{sec:selfcal}).  The total dynamic range,
measured as the peak brightness in
Sgr B2 to the RMS noise in a signal-free region of the combined 7m+12m image,
is 18000 (noise $\sim0.09$ mJy/beam), while the dynamic range within one
primary beam ($\sim0.5$\arcmin) of Sgr B2M is only 5300 (noise $\sim0.3$
mJy/beam).  Because of the dynamic range limitations and an empirical
determination that clean did not converge if allowed to go too deep, we cleaned
to a threshold of 0.1 mJy/beam over all pixels with $S_\nu > 2.5$ mJy bm$^{-1}$
as determined from a previous iteration.
% We performed this same
% process for both the longest-baseline data only (resolution $\sim0.5\arcsec$,
% largest angular scale theoretically 15\arcsec\ [the shortest baseline] but more
% practically $\sim7$\arcsec\ [the 5th percentile baseline length]) and the
% merged 7m + two 12m configuration data.  The merged data are more useful for
% studying extended structures but have lower dynamic range, while the
% long-baseline-only data are excellent for extracting and analyzing pointlike or
% compact sources.

We also produced cubes of all of the spectral lines.  These were lightly
cleaned with a maximum of 2000 iterations of cleaning to a threshold of
100 mJy/beam.  No self-calibration was applied.
Before continuum subtraction, dynamic range related artifacts similar to those
in the continuum images were present, but these structures are identical across
frequencies, and were therefore removable in the image domain.  We use
median-subtracted cubes for the majority of our analysis, noting that the only
location in which an error on the continuum $>5\%$ is expected is the Sgr B2
North core \citep{Sanchez-Monge2017a}.

\FigureTwo
{figures/continuum_peak_DeepSouth_saturated.pdf}
{figures/cores_on_continuum_peak_DeepSouth_saturated.pdf}
{Images of the 3 mm continuum in the Sgr B2 DS region.  The right figure
includes a red dot at the position of each identified continnuum pointlike
source.  The \hii region Sgr B2 S is the bright source at the top of the image;
imaging artifacts can be seen surrounding it.  The largest angular
scales are noisier than the small scales; the $\sim20\arcsec$-wide east-west
ridge at around -28:24:30 is likely to be an imaging artifact.  By contrast,
the diffuse components in the southern half of the image are likely to be real.
}
{fig:continuumDS}{1}{0.5\textwidth}



\subsection{Column Density Maps}
We use archival data from SCUBA, SHARC, and Herschel to create column density
maps.  We combined the SHARC and SCUBA data with Herschel SPIRE 350 and 500 \um
images \citep{Molinari2010a}, respectively.  The data combination is discussed
in detail in Appendix \ref{sec:singledishcomb}.  

The SHARC data were reported in \citet{Bally2010a} and have a nominal
resolution of 9\arcsec at 350 \um, however, at this resolution,
the SHARC data display a much higher surface brightness than the Herschel
data on a similar scale.  A resolution of 11.5\arcsec gives a better
surface brightness match and is consistent with the measured scale of Sgr
B2 N in the image.  This calibration difference is likely to have been a
combination of blurring by pointing errors, surface imperfections, and
the gridding process, and by flux calibration errors.  In any case, the
Herschel data provide the most trustworthy absolute calibration scale.

The SCUBA 450 \um data were reported in \citet{Pierce-Price2000a} and
\citet{di-Francesco2008a} with a resolution of 8\arcsec.  We found that the
SCUBA data had a flux scale significantly discrepant from the Herschel data,
even accounting for the central wavelength difference.  We had to scale the
SCUBA data up by a factor $\approx3.0$ to make the data agree with the Herschel
images on the angular scales they are both hypothetically sensitive to.
While such a large flux calibration error seems implausible, the measured
FWHM is approximately 14\arcsec, which means the beam area is $\approx3\times$
larger than the theoretical size.  We attempted to fit several other isolated
sources in the large SCUBA map, and the smallest FWHM we measured was $\approx10.5$
arcsec.  Between the larger beam area, flux calibration errors \citep[quoted at
20\percent in][]{Pierce-Price2000a}, and the dust emissivity correction
(35-50\percent for $\beta=3-4$), this large flux scaling factor is actually
plausible.  


To determine the column density, we adopted a few independent approaches.
First, we use the Herschel data to perform SED fits to each pixel (Battersby et
al, in prep).  We performed these fits at 25\arcsec resolution, excluding the
500 \um channel.  To obtain higher resolution column density maps, we used the
combined Herschel-SHARC and Herschel-SCUBA maps assuming optically thin dust
using both a constant temperature and the temperature measured with Herschel at
25\arcsec resolution interpolated onto the higher-resolution SCUBA and SHARC
grids.  Because of the interpolation or fixed temperature assumptions, the
column maps are not very accurate and should not be used for systematic
statistical analysis of the column distribution.


\section{Analysis}

\subsection{Continuum Source Identification}
\label{sec:contsources}
We selected continuum point sources as candidate cores or protostars by eye.
An automated selection is not viable across the majority of the field because
there are many extended \hii regions that dominate the overall map emission.  A
future automated selection algorithm may work if images at comparable
resolution at other frequencies become available; the \hii-region sources could
then be excluded.  Additionally, however, there are substantial imaging
artifacts produced by the extremely bright emission sources in Sgr B2 M ($S_{3
mm,max} \approx 1.6$ Jy) and Sgr B2 N ($S_{3 mm,max} \approx 0.3$ Jy) that make automated
source identification particularly challenging in the most source-dense regions.
All of these features are evident in Figures \ref{fig:continuumMandN} and
\ref{fig:continuumDS}.

Because the noise varies significantly across the map, a uniform selection
criterion is not possible.  We therefore include two levels of source
identification, `high confidence' sources, which are selected conservatively in
regions of low-background, and `low-confidence' sources that are somewhat lower
signal-to-noise and are often in regions with higher background.  Both of
these selection criteria are significantly more conservative than a local
$5-\sigma$ threshold.

We measure the local noise for each source by taking the median absolute
deviation in an annulus 0.5 to 1.5 \arcsec around the source center.
All but 7 sources have signal-to-local-noise ratios $S/N>7$.  These
sources are all in regions of particularly high background or source
density and therefore have overestimated local noise.

Our selection criteria result in a reliable but potentially incomplete catalog.

For a subset of the sources, primarily the brightest, we measured the spectral
index $\alpha$.  We relied on CASA's $\alpha$ and $\sigma(\alpha)$ maps to
obtain these measurements, including only those with $|\alpha| > 5
\sigma(\alpha)$ or $\sigma(\alpha) < 0.1$.  Several of the brightest sources
did not have significant measurements of $\alpha$ because they are in the
immediate neighborhood of Sgr B2 M or N and therefore have significantly higher
background and noise, preventing a clean measurement.

\section{Results}

We detected \ncores compact continuum sources.  Their flux distribution is
shown in Figure \ref{fig:fluxhist}.  The distribution of their measured
spectral indices $\alpha$ is shown in Figure \ref{fig:alphahist}.


\subsection{Source Classification}
\label{sec:classification}
For the majority of the detected sources, we have only a continuum detection at
3 mm.  No lines are detected peaking toward most of the sources, especially the
faint ones.  A subset have detections at other bands and can be classified
based on previous literature work, especially those associated with \hii
regions detected at 0.7 and 1.3 cm
\citep{Gaume1995a,Mehringer1995b,de-Pree1996a,Pree2015a}.  In this section, we
employ various means to classify the sample of new sources.

% core_flux_distributions
\FigureOneCol{figures/core_peak_fluxdensity_powerlawfit.png}
{A histogram of the peak flux density of the observed sources with a powerlaw
fit to the data excluding known \hii regions.  The powerlaw is an excellent
fit, and it is shallower than the IMF slope, with $\alpha=1.94\pm0.07$.  The
two brightest regions not identified as \hii regions are Sgr B2M f1 and Sgr B2N
K2; in these regions, it is unclear whether the free-free or dust emission
dominates.}
{fig:fluxhist}{1}{0.5\textwidth}

We first note some key properties of dust at 3 mm.   At 8.4 kpc, a 1 mJy source
corresponds to an optically thin dust mass of $M(40\mathrm{K})=18$ \msun or
$M(20\mathrm{K})=38$ \msun assuming a dust opacity index $\beta=1.75$
($\alpha=3.75$ on the Rayleigh-Jeans tail) to
extrapolate the \citet{Ossenkopf1994a} opacity to $\kappa_{3mm}=0.0018$ cm$^2$
g$^{-1}$.  Our dust-only 5-$\sigma$ sensitivity limit at 40 K therefore ranges
from $M>7$ \msun (0.5 mJy) to $M>45$ \msun (2.5 mJy) across the map.  If we
were to assume that these are all cold, dusty sources, as is typically (and
reasonably) assumed for local clouds, they would be extremely massive and
dense, with the lowest measurable density being $n(40\mathrm{K}) > 3\ee{6}$
\percc (corresponding to 7 \msun in a 0.5\arcsec radius sphere).  Such extreme
objects are possible, but since we have detected $>100$ of these sources, we
evaluate other possibilities.


\subsubsection{Alternative 1: The sources are externally ionized gas blobs}
One possibility is that these sources are not dusty at all, nor pre- or
protostellar, but are instead the brightest compact clumps surrounding \hii
regions.  They would then be analogous to the heads of cometary clouds,
externally ionized globules (``EGGs"), or proplyds, and their observed emission
would give no clue to their nature because the light source is extrinsic.

The majority of the detected sources have size $<4000$ AU, i.e., they are
unresolved.  By contrast, the free-floating EGGs so far observed have sizes
10,000-20,000 AU \citep{Sahai2012a,Sahai2012b}, so they would be resolved in
our observations.  Toward the brightest frEGG in Cygnus X, \citet{Sahai2012b}
measured a peak intensity $S_{8.5 GHz} \approx 1.5$ mJy/beam in a
$\approx3\arcsec$ beam.  Cygnus X is $6\times$ closer that the Galactic center,
so their beam size is the same physical scale as ours.  If the free-free
emission is thin, the brightness in our data would be $S_{95 GHz} =
(95/8.5)^{-0.1} S_{8.5 GHz} = 0.79 S_{8.5 GHz} \approx 1.2$ mJy/beam.  These
frEGGs would be detectable in our data.  Comparison to radio observations
at a comparable resolution will be needed to rule out the externally ionized
globule hypothesis for resolved regions.


\FigureOneCol{figures/core_alpha_coloredbyclass.png}
{A histogram of the spectral index $\alpha$ for those sources with a statistically
significant measurement.  The \hii regions cluster around $\alpha=0$, as expected
for optically thin free-free emission, while the unclassified sources cluster
around $\alpha=3$, which is weakly consistent with dust emission.
There are two particularly notable outliers: the source with a highly negative
spectral index, Source 167, has a statistically significant measurement but is
in a region of particularly high extended and diffuse background, so the
measurement may not be reliable.  The other unidentified source with $\alpha\sim-1$
is Source 80, which is very close to the \hii region Sgr B2 S, but is not detected
in the \citet{de-Pree1996a} 1.3 cm data; it may be an \hii region or it may
be contaminated by the nearby \hii region.}
{fig:alphahist}{1}{0.5\textwidth}

If the detected sources were either EGGs or cometary clouds, we would expect
them to be located within \hii regions.  Many of the sources are near \hii
regions, as seen in Figure \ref{fig:coreson20cmandhc3n}a.  However, they are nearly all
associated with a ridge of HC$_3$N emission (Figure \ref{fig:coreson20cmandhc3n}b).  If
they are deeply embedded within the molecular material, they cannot be
externally ionized.  The current data do not provide enough information on the
geometry of the clouds to rule out the possibility that the point sources are
just illuminated cloud edges, but the fact that the ionized gas is brightest
adjacent to, rather than on top of, the HC$_3$N suggests that the HC$_3$N
traces a full molecular cloud rather than a thin PDR-like layer.

A final point against the externally ionized hypothesis is the observed
spectral indices shown in Figure \ref{fig:alphahist}.  We measured spectral
indices for 43 sources, of which 23 have $\alpha>2$.  These sources are
inconsistent with free-free emission and are at least reasonably consistent
with dust emission.  Of the 20 that are consistent with free-free emission, 11
are known \hii regions, hinting that our sample is dominated by dusty objects,
not externally ionized objects.

%This
%scale is consistent with that of frEGGs seen in Carina \citep[e.g.][]{Sahai2012a}.
%The flux density of our sources is 1-2 orders of magnitude below what we would
%predict from the dust observed in the \citet{Sahai2012a} object, though the
%free-free... \todo{Look up observed free-free emission from proplyds and frEGGs.
%Could we have frEGGs?}

% The `tadpole' \citep{Sahai2012b} has $S_{22 \mathrm{GHz}} = 30$ mJy at d=1.4 kpc, resolved
% to $\sim10\times10$ arcsec.  In the CMZ... this would be visible.  Maybe compare to Zadeh's
% recent obs of "proplyds"?

% The key requirement, if these are all indeed externally ionized molecular
% blobs, is that of a strong ionizing radiation field.  Indeed, L-band
% observations \citep{Yusef-Zadeh2004a} show that many of these sources are
% located along the outer edges of a large-scale \hii region (Figure
% \ref{fig:coreson20cm}).
% 
% % However, their location poses a problem for the
% % frEGG/proplyd hypothesis: % this next sentence doesn't really make sense...
% %these sources are observed \emph{within} \hii
% %regions, not along their outskirts. 
% However, when analogous structures are present
% along the outskirts of clouds, they are usually accompanied by a larger
% contiguous cloud edge, which results in a continuous sharp-edged bright feature
% corresponding to a PDR.  Such edges are seen in the Orion Bar and M16's
% ``Pillars of Creation''.   We do not observe any such features here.
% 
% We know from our molecular line observations in HC$_3$N that most of the
% continuum sources lay along a molecular ridge (TODO: FIGURE), so it appears
% most of these sources are embedded in molecular material.

% Nice idea, but h41a is just too weak
% TODO: H41a TE peak map.  Show locations relative to confirmed HII regions

%% pointsource_overlay
%\FigureOneCol{figures/cores_on_20cm_continuum.png}
%{The location of the detected continuum sources (red points) overlaid on a 20
%cm continuum VLA map highlighting the diffuse free-free (or possibly
%synchrotron) emission in the region \citep{Yusef-Zadeh2004a}.}
%{fig:coreson20cm}{1}{0.5\textwidth}
%
%% pointsource_overlay
%\FigureOneCol{figures/cores_on_HC3N_peak.png}
%{The location of the detected continuum sources (red points) overlaid on a map
%of the HC$_3$N peak intensity.  HC$_3$N traces moderate-density molecular gas.}
%{fig:coresonhc3n}{1}{0.5\textwidth}

\FigureTwo
{figures/cores_on_20cm_continuum.png}
{figures/cores_on_HC3N_peak.png}
{(left) The location of the detected continuum sources (red points) overlaid on a 20
cm continuum VLA map highlighting the diffuse free-free (or possibly
synchrotron) emission in the region \citep{Yusef-Zadeh2004a}.
(right) Continuum sources overlaid on a map
of the HC$_3$N peak intensity.  HC$_3$N traces moderate-density molecular gas.}
{fig:coreson20cmandhc3n}{1}{0.5\textwidth}

\subsubsection{Alternative 2: The sources are \hii regions produced by
interloper ionizing stars}
If there is a large population of older, but still ionizing, stars, they could
ignite \hii regions when they fly through molecular material.  See
\ref{sec:theyarehiiregions} for calculations of stationary \hii region
properties.  The main problem with this scenario is the spatial distribution of
the observed sources.  While most of the continuum sources are associated with
dense gas and dust ridges, not all of the high-column molecular gas regions
have such sources in them (i.e., the left and right sides of the image in
Figure \ref{fig:coresonhc3n}, where molecular material is seen with no
associated millimeter sources).  If there is a free-floating population of OB
stars responsible for the 3 mm point source population, their distribution
should match that of the gas.  Also, there is no such population of sources
seen outside of the dense gas in the infrared \citep[TODO: Who has done
infrared studies of Sgr B2?  You can infer what I have stated `by inspection'
of 2MASS, but it would be more straightforward to quote someone else][]{},
which again we should expect if there is a uniformly distributed population.  
Finally, the spectral indices discussed above (Figure \ref{fig:alphahist})
suggest the previously-unidentified sources are dust emission sources,
not free-free sources.



\subsubsection{Alternative 3: The sources are \hii regions produced by
recently-formed OB stars}
\label{sec:theyarehiiregions}

For an unresolved spherically symmetric \hii region ($R=4000$ AU), the expected
flux density is $S_{95 \mathrm{GHz}} = 4.7$ mJy for a $Q_{lyc}=10^{47}$ \pers
source (assuming $T_e=7000$ K), and that value scales linearly with $Q_{lyc}$
as long as the source is optically thin.  Rearranging \citet{Condon2007a} equations
4.60 and 4.61:
\begin{eqnarray}
S_{\nu}(Q_{lyc})  = 4.67 \left[1-\exp\left(c_* T_* \nu_* EM_* \right) \right] \nonumber \\
\nu_* = \left(\frac{\nu}{\mathrm{GHz}}\right)^{-2.1} \nonumber \\
T_* = \left(\frac{T_e}{10^4 \mathrm{K}}\right)^{-1.35} \nonumber \\
c_* = -3.28\times10^{-7} \nonumber \\
EM_* = \frac{3 Q_{lyc}}{4 \pi R^2 \alpha_b} \nonumber \\
\end{eqnarray}
where $\alpha_b=2\ee{-13}$ cm$^3$ s$^{-1}$, $Q_{lyc}$ is the count rate
of ionizing photons in $s^{-1}$, and $R$ is the \hii region radius.

An extremely compact \hii region,
e.g., one with $R<100$ AU and corresponding density $n>10^6$ \percc, would be
optically thick and therefore fainter, $S_{95 \mathrm{GHz}}(R=100 \mathrm{AU},
Q_{lyc}=10^{47} \pers)=3.4$ mJy.  Even the brightest O-stars could produce \hii
regions as faint as 0.5 mJy if embedded in extremely high density gas; above
$Q_{lyc}>10^{47}$ \pers, a 25 AU \hii region would be $\sim0.5$ mJy.

Figure \ref{fig:hiibrightness} shows the predicted brightness for various \hii
regions produced by OB stars and the density required for those \hii regions
to be the specified size.  In
order for the detected sources to be O-star-driven \hii regions, with $10^{47}
< Q_{lyc} < 10^{50}$ \pers, they must be optically thick and therefore
extremely compact and dense.  There is a narrow range of late O/early B stars,
$10^{46} < Q_{lyc} < 10^{47}$ \pers, that could be embedded in compact \hii
regions of almost any size and produce the observed range of flux densities.
Anything fainter, later than $\sim$B0 ($Q_{lyc}<10^{46}$ \pers), would be
incapable of producing the observed flux densities.
Any brighter stars would have to be embedded in dust that, at 40 K, would
outshine the \hii region; more likely, such sources would have much hotter dust
and therefore would be much brighter (and more extended) than our observations
allow.

This restrictive parameter space, combined with a steep luminosity function
that implies there are many more sources at slightly lower luminosity, is
evidence against the population being dominated by \hii regions.  The spectral
indices also support this conclusion.



\Figure{figures/HII_region_brightness.png}
{Simple models of spherical \hii regions to illustrate the observable
properties of such regions.  The \hii region size is shown by line color; the legend
in the left plot applies to both figures.  (left) The expected brightness temperature (left
axis) and corresponding flux density within a FWHM=0.5\arcsec beam (right axis) as a
function of the Lyman continuum luminosity for a variety of source radii.
(right) The density required to produce an \hii region of that radius.  The
horizontal dashed line shows the density corresponding to an unresolved dust
source at the 5-$\sigma$ detection limit ($\approx0.5$ mJy, or about 10 \msun
of dust,
assuming $T=40$ K).  Above this line, dust emission would dominate over
free-free emission.  The dotted line shows the density required for dust
emission to produce a 10 mJy source at $T=40$ K.  
As seen in the left plot, for any moderate-sized \hii region, $R>100$ AU, a high-luminosity
star ($Q_{lyc} > 10^{47}$ \pers) would produce an \hii region brighter than the
majority of our sample, which includes only a few sources brighter than 10 mJy.
The densities required to produce \hii regions within our observed range
($1<S_\nu<10$ mJy) are fairly extreme, $n\gtrsim10^6$ \percc, for O-stars.}
{fig:hiibrightness}{1}{18cm}


\subsubsection{The sources are protostars}
\label{sec:theyareprotostars}
After ruling out the other hypotheses, we conclude that these sources are
predominantly embedded protostars.  Their emission is likely dust-dominated,
but is probably warmer than the cloud average $\sim20-40$ K.
We test and validate the hypothesis that most or all of the sources
are protostellar in this section.

We cross-matched out source catalog with catalogs of \hii regions and 
methanol masers.  Class II methanol masers are always associated
with sites of high-mass star formation.
The \citet{Caswell2010a} Methanol Multibeam Survey identified 11 sources in our
observed field of view, of which 10 have a clear match in our catalog.
Several other sources in our catalog match known \hii regions from
\citet{Gaume1995a}, mostly associated with the brightest sources in our sample;
these all have $S_{3 \textrm{mm}} > 9$ mJy.  Additionally, some of the
sources have X-ray counterparts in the \citet{Muno2009a} Chandra point source
catalog.


We compare our detected sample to that of the Herschel Orion Protostar Survey
\citep[HOPS;][]{Furlan2016a} in order to get a general sense of what types of
sources we have detected.  We selected this survey for comparison because it is
one of the largest protostellar core samples with well-characterized bolometric
luminosities available.
Figure \ref{fig:hopshist} shows the HOPS source
fluxes at 870\um scaled to 3 mm assuming a dust opacity index $\beta=1.5$,
which is shallower than usually inferred, so the extrapolated
fluxes may be slightly overestimated.  The 870\um data were acquired with a
$\sim20\arcsec$ FWHM beam, which translates to a resolution $\sim1\arcsec$ at
$d_{Sgr B2} = 8.4$ kpc assuming $d_{Orion}=415$ pc, so our beam size is somewhat smaller than
theirs.  The HOPS sources are all fainter than the Sgr B2 sources.  The
brightest HOPS source, with $L_{tot}<2000$ 
\lsun, would only be 0.2 mJy in Sgr B2, or about a 4-$\sigma$
source - below our detection threshold even in the noise-free regions of the
map.  We  conclude that the Sgr B2 sources are much more luminous
and are therefore massive protostars.

% HOPS_core_plots
\FigureOneCol{figures/core_peak_intensity_histogram_withHOPS.png}
{A histogram combining the detected Sgr B2 cores with predicted flux densities
based on the HOPS \citep{Furlan2016a} survey.  The HOPS histogram shows the 870
\um data from that survey scaled to 3 mm assuming $\beta=1.5$.  Every HOPS
source is well below the detection threshold for our observations.}
{fig:hopshist}{1}{0.5\textwidth}

This conclusion is supported by a more direct comparison with the Orion nebula
as observed at 3 mm with MUSTANG \citep[][Figure
\ref{fig:orioncompare}]{Dicker2009a}.  Their data were taken at
9\arcsec FWHM resolution, corresponding to 0.48\arcsec at $d_{Sgr B2}$.  The
peak flux density measured in that map is toward Source I, $S_{90 GHz}(d_{Sgr
B2}) = 3.6$ mJy.  Source I would therefore  be detected and would be
somewhere in the middle of our sample.  It is  extended, and the
extended component would be readily detected in our data. 
Since Source I is the only known high-mass YSO in the Orion cloud, and it would
be detectable while no other sources in the Orion cloud would be, it appears
safe to conclude that all of our detected sources are MYSOs.
%The M42 nebula would
%also be readily detectable, and is very similar in size and surface brightness
%to the \hii region Sgr B2 \hii T.

% Bontemps+ 2010: 3.5mm flux of N63 ~ 36 mJy, -> 1 mJy @ Sgr B2

\FigureOneCol{figures/Orion_SgrB2HII_side_by_side.png}
{Comparison of two extended \hii regions in Sgr B2 to the M42 nebula in Orion.
The three panels are shown on the same physical and color scale assuming
$d_{Orion} = 415$ pc and $d_{Sgr B2} = 7.8$ kpc and that the ALMA and MUSTANG
data have the same continuum bandpass.  Sgr B2 \hii T is comparable in
brightness and extent to M42; Sgr B2 \hii L is much brighter and is saturated
on the displayed brightness scale.  The compact source to the top right of the
M42 image is Orion Source I; the images demonstrate that Source I and the entire
M42 nebula would be easily detected in our data.
}
{fig:orioncompare}{1}{0.5\textwidth}

While we have concluded that the sources are dusty, massive protostars, the
spectral indices we measured are somewhat surprising.  Typical dust clouds in
the Galactic disk have dust opacity indices $\beta\sim1.5-2$
\citep{Schnee2010a,Shirley2011a,Sadavoy2016a}.  Our spectral indices are lower
than these, with only 3 sources having measured $\beta=\alpha-2 > 1.5$ (at the
$2\sigma$ level, up to 11 sources are consistent with $\beta=1.5$, but this is
primarily because of their high measurement error).  A shallower $\beta$
implies free-free contamination, large dust grains, or optically thick surfaces
are present within our sources.  Since the above arguments suggest that the
sources are high-mass protostars, the free-free contamination and optically
thick inner region models are both plausible.

\subsection{Source distribution functions and the star formation rate}

The flux density distribution of the non-\hii region sources follows a powerlaw
with slope $\alpha=1.94\pm0.07$ \citep[fitted with the MLE method
of][]{Clauset2007a}.  If we assume that the stellar mass is linearly
proportional to the 3 mm continuum flux density, this measurement implies a
slope shallower than the $\alpha\sim2.35$ expected for a normal IMF.  It is
possible that the IMF is genuinely different from Salpeter here, but it is more
likely that the more massive stars are surrounded by warmer gas, implying that
the source mass distribution is steeper than the source flux distribution.

If we make the very simplistic assumptions that the sources we detect are all
$L\gtrsim2000$ \lsun ($M\gtrsim8 \msun$), we can infer the total (proto)stellar
mass in the observed region.  Using a \citet{Kroupa2001a} mass function with
$M_{max}=200$ \msun, 23\% of the mass is contained in $M>8\msun$ stars.  Using
$M=8\msun$ as the lower-limit case for each source, the identified sources have
total mass $M(>8)=1800\msun$.  The total stellar mass implied is $M_{tot} =
8\ee{3}$ \msun.  If instead we assume each source has a mass equal to the mean
stellar mass for $M>8$ \msun, $\bar{M}=21.1$ \msun, then the total inferred
stellar mass is $M_{tot}=2\ee{4}$ \msun.  These are lower limits in the Sgr B2
N and M regions because our catalog is incomplete due to confusion and dynamic
range limitations.  Additionally, we are using a single-star IMF and our
resolution is only $\sim4000$ AU, so it is likely that we have undercounted by
$\gtrsim2\times$, since high-mass stars have a high multiplicity fraction
\citep{Mason2009a}.

% COMPARE TO SCHMIEDEKE

For each subcluster, we count the number of \hii regions identified in our
survey plus those identified in previous works \citep{de-Pree1996a} and
we count the number of protostellar cores not associated with \hii regions.
To estimate the stellar mass, we assume each core contains a star
with $\bar{M(8-20)} = 12$ \msun and each \hii region contains a $\geq$B0 star
with $\bar{M(>20)} = 45$ \msun.  In Table \ref{tab:clustermassestimates},
this estimate is shown as $M_{obs}$.  We also compute the total stellar
mass using the mass fractions $f(M>20) = 0.14$ and $f(8<M<20)=0.09$.
The inferred masses computed from \hii region counts and from core
counts are shown in columns $M_{inferred,\hii}$ and $M_{inferred,cores}$
respectively; $M_{inferred}$ is the average of these two estimates.
If our assumptions are correct and the mass distribution is governed
by a power-law IMF, we expect $M_{inferred,\hii} = M_{inferred,cores}$.
Except for Sgr B2 M, the core-based and \hii-region based estimates agree
to within $\sim25\%$, which is about as good as expected from Poisson
noise in the counting statistics.  Sgr B2 M has the largest sample in both
counts and has a factor of two discrepancy; this is likely because of the
combined effects of source confusion at our 0.5\arcsec resolution and the
increased noise around the extremely bright central region that makes detection
of $<2$ mJy sources impossible.

We compare our mass estimates to those of \citet{Schmiedeke2016a}, who inferred
stellar masses primarily from \hii region counts.  The last two columns of Table
\ref{tab:clustermassestimates} show the observed and estimated masses based on
\hii region counts.  For Sgr B2 M and N, our results are similar, as expected
since our catalogs are identical.  For S and NE, we differ by a large factor:
TODO: I don't know why; we have effectively the same source counts (though they
have 1 more in South than I do) but significantly different inferred masses.

% cluster mass estimates table
% stellar_mass_estimates.py
\begin{table*}[htp]
\centering
\caption{Cluster Masses}
\begin{tabular}{cccccccccc}
\label{tab:clustermassestimates}
Name & $N(cores)$ & $N(H\textsc{ii})$ & $M_{count}$ & $M_{inferred}$ & $M_{inferred, H\textsc{ii}}$ & $M_{inferred, cores}$ & $M_{count}^s$ & $M_{inf}^s$ & SFR \\
 &  &  & $\mathrm{M_{\odot}}$ & $\mathrm{M_{\odot}}$ & $\mathrm{M_{\odot}}$ & $\mathrm{M_{\odot}}$ & $\mathrm{M_{\odot}}$ & $\mathrm{M_{\odot}}$ & $\mathrm{M_{\odot}\,kyr^{-1}}$ \\
\hline
M & 20 & 51 & 2600 & 9700 & 17000 & 2700 & 1295 & 20700 & 13 \\
N & 12 & 7 & 460 & 2000 & 2300 & 1600 & 150 & 2400 & 2.6 \\
NE & 4 & 2 & 140 & 600 & 650 & 540 & 52 & 1200 & 0.81 \\
S & 5 & 1 & 110 & 500 & 330 & 680 & 50 & 1100 & 0.68 \\
Total & 235 & 72 & 6100 & 28000 & 24000 & 32000 & 1993 & 33400 & 37 \\
\hline
\end{tabular}
\par
$M_{count}$ is the mass of directly counted protostars, assuming each millimeter source is 12.0 \msun, or 45.5 \msun if it is also an \hii region.  $M_{inferred,cores}$ and $M_{inferred,\hii}$ are the inferred total stellar masses assuming the counted objects represent fractions of the total mass 0.09 (cores) and 0.14 (\hii regions).  $M_{inferred}$ is the average of these two.  $M_{count}^s$ and $M_{inf}^s$ are the counted and inferred masses reported in \citet{Schmiedeke2016a}.  The star formation rate is computed using an age $t=0.74$ Myr, which is the time of the last pericenter passage in the \citet{Kruijssen2015a} model.  The \emph{total} column represents the total over the whole observed region.    The clusters sum to much less than the \emph{total} because the Deep South region is not included, and it dominates the overall core count.
\end{table*}




% While we identified \ncores sources from the continuum data, since we have only
% a single continuum band available, it is difficult to classify most of these
% except to say that they are certainly forming or recently formed stars.
% However, for a small subset, we have spectral line detections in either
% molecular or ionized species that tell us qualitatively whether a source is
% ionizing an \hii region or is surrounded by interesting molecular species.
% 
% \todo{Continue here - give the subset of sources with good line IDs (which is
% probably a by-hand process) and show example spectra of something not Sgr B2 M
% or N}

\subsection{An examination of star formation thresholds}
Many authors \citep[e.g.,][]{Lada2010a} have proposed that star formation can
only occur above a certain density or column density threshold\footnote{Column
density is more commonly used because of its observational convenience, but it
is physically meaningless unless high column density leads to high optical
depths and thereby changes the gas's ability to cool.}. 
We discuss our measurements of column thresholds in this Section.



\subsubsection{Comparison to Lada, Lombardi, and Alves 2010}
In this section, we compare the star formation threshold in Sgr B2 to that in
local clouds performed by \citet{Lada2010a}.  They determined that all star
formation in local clouds occurs above a column density threshold $M_{thresh} >
116$ \msun pc$^{-2}$, or $N_{thresh}(\hh) > 5.2\ee{21}$ \persc assuming the
mean particle mass is 2.8 amu \citep{Kauffmann2008a}.  We first note, then,
that \emph{all pixels} in our column density maps are above this threshold
by \emph{at least} a factor of 10.

However, Sgr B2 is \dsgrb kpc away from us in the direction of our Galaxy's
center, meaning there is a potentially enormous amount of material unassociated
with the Sgr B2 cloud along the line of sight.  This material may have column
densities as low as
5\ee{21} \persc or as high as 5\ee{22} \persc, as measured from relatively
blank regions in the Herschel column density map \citep{Battersby2011a}.  The
former value corresponds to
the background at high latitudes, $b\sim0.5$, while the latter  is
approximately the lowest seen within our field of view. 
%If the higher value is the 
%correct foreground, there must be a perfect vacuum surrounding the dense gas in
%the Sgr B2 cloud - the `hole' seen in Fig...TODO: show a hole figure.
Even with the very aggressive foreground value of 5\ee{22} \persc subtracted,
nearly the whole Sgr B2 cloud exists above this threshold.
%We can therefore
%immediately rule out the possibility that there is a universal star formation
%column threshold, since a large fraction of the observed volume exhibits
%no hint at all of star formation activity.
%-What kind of stars are we sensitive to?  Are they?
%(this is now handled above)

To directly compare our observations to the star formation thresholds reported
in \citet{Lada2010a}, we examined the column density associated with each
millimeter continuum source.  The \citet{Lada2010a} data used a variable
resolution for the column measurements toward their sample, ranging from
0.06-0.35 pc (equivalent to 1.5 to 9.2 \arcsec at a distance of \dsgrb).  The
Herschel data we have available with per-pixel SED fits lack the resolution
needed to make a direct comparison to the Lada et al data set, but the SHARC
and SCUBA data have resolution approximately equivalent to that used in the
Orion molecular cloud in their survey.  We therefore use a range of temperatures
bracketing the observed range in the Herschel maps ($\sim20-50$ K) to produce
column density maps from the SCUBA and SHARC data.  Figure
\ref{fig:corebackgroundcdf} shows the cumulative distribution function of the
column density associated with each identified continuum source; the column
density used is the nearest-neighbor pixel to the source in the column density
maps.  Even using the conservative maximum temperature $T_{dust}=50$ K
(resulting in the minimum column density), all of the sources exist at a column
density an order of magnitude higher than the Lada threshold, and they exist
above that threshold even if the foreground is assumed to be an extreme $5\ee{22}$
\persc.


The \citet{Lada2010a} sample used Spitzer observations of nearby clouds that
were nearly complete to stars at least as small as 0.5 \msun.  By contrast, as
discussed in Section \ref{sec:theyareprotostars}, our survey is sensitive only
to stars with $M\gtrsim8$ \msun.  The apparently higher threshold either means
that there is a genuinely higher threshold for star formation in the CMZ or
that there is a higher threshold for high-mass star formation that may still
be universal.


% core_local_brightness
\FigureOneCol{figures/core_background_column_cdf.png}
{Cumulative distribution functions of the background column density associated
with each identified 3 mm continuum source.  The column densities are computed
from a variety of maps with different resolution and assumed temperature.
The Herschel maps use SED-fitted temperatures \citep{Battersby2017a} at
25 \arcsec resolution (excluding the 500 \um data point) and 36 \arcsec resolution.
The SHARC 350 \um and SCUBA 450 \um maps both have higher resolution ($\sim10\arcsec$)
but no temperature information; we used an assumed $T_{dust}=20$ and $T_{dust}=50$ K
to illustrate the range of possible background column densities (hatched
red and blue).  The thick solid red and blue lines show the SHARC and SCUBA column
density images using Herschel temperatures interpolated onto their grids: these
curves are closer to the 20 K than the 50 K curve and serve as the best estimate
column density maps.  The SHARC data fail to go to a cumulative fraction of 1
because the central pixels around Sgr B2 M and N are saturated (the lower temperature
assumptions result in optical depths $>1$, which cannot be converted to column
densities using the optically thin assumption).  The vertical
dashed line shows the $N(\hh)=5.2\ee{21}$ column density threshold from
\citet{Lada2010a}.}
{fig:corebackgroundcdf}{1}{0.5\textwidth}

% plot_codes/pointsource_overlay_scuba_column.py
\FigureOneCol{figures/cores_on_SCUBA_column_saturated.png}
{Overlay of the core locations on the SCUBA column density map
created by interpolating the Herschel-measured temperatures onto the SCUBA grid
and assuming the dust is optically thin at 450 \um.  The square grid at the center
shows a region affected by saturation in the Herschel data.
Contours are shown at $N(\hh) = 2\ee{23}$ \persc (green dashed lines)
and $N(\hh) = 5,10,50,100\ee{23}$ \persc (blue lines).
As shown in Figure \ref{fig:corebackgroundcdf}, the threshold above which nearly
all cores are found is high, but this figure shows that the core density is not
well-correlated with the column density: there is a relative dearth of cores
in the high-column north region and an overabundance in the moderate-column deep
south region.
}
{fig:coresonscubacol}{1}{0.5\textwidth}

% \Figure{figures/flux_histograms_with_core_location_CDF.png}
% {Histograms of the brightness measured with a variety of instruments at
% different submillimeter bands with the cumulative distribution function (CDF)
% of the \emph{background} brightness surrounding each core superposed.  The
% X-axis units are arbitrary (because right now I don't know the units of all of
% these) except for column, which is in units of cm$^{-2}$ of \hh as derived from
% SED fits to Herschel data (Battersby+).  The grey line is of the observed
% region in Sgr B2 and the blue line is of G0.253+0.016.  The thick grey line is
% the CDF of core background brightness, and is labeled by the right axis.}
% {fig:fluxhist}{1}{\textwidth}

\subsubsection{Comparison to other CMZ clouds}
In G0.253+0.016 (The Brick, G0.253), very little star formation
has been observed \citep{Longmore2013a,Johnston2014a,Rathborne2015a} despite
most of the cloud existing above the locally measured \citet{Lada2010a} column
density threshold.  The column density distribution function for G0.253
is shown in Figure \ref{fig:bricksgrb2colcompare}.

Comparing Sgr B2 to G0.253, the majority of the Sgr B2 cloud is at higher
column than G0.253.  The presence of star formation in Sgr B2 nearly all occurs
at a higher column than exists within G0.253 (Figure
\ref{fig:bricksgrb2colcompare}).  The lack of observed cores in The Brick is
therefore consistent with the active SF seen in Sgr B2.

% core_local_brightness
\FigureOneCol{figures/compare_brick_sgrb2_colPDF.png}
{Histograms of the column density of G0.253+0.016 (blue) and Sgr B2 (gray)
using the combined SCUBA 450 \um and Herschel 500 \um intensity with the
interpolated Herschel dust temperatures.  The cumulative distribution of
core `background' column densities in Sgr B2 is shown as a thick gray
line.}
{fig:bricksgrb2colcompare}{1}{0.5\textwidth}

\section{Discussion}
We have reported the detection of a large number of point sources and inferred
that they are most likely all high-mass protostars.  These sources universally
reside in gas above $N\gtrsim2\ee{23}$ \persc gas.  In this section, we discuss
the implications of this apparent threshold for high-mass star formation in the
CMZ.

A theoretical threshold for high-mass star formation, $N > 1$ g \persc was
developed by \citet{Krumholz2008a}.  Since all of the sources we have detected
reside above this threshold and we determined our sources are all likely to be
massive protostars in Section \ref{sec:theyareprotostars}, we have apparently
confirmed this threshold.


While there is an apparent column density threshold required for high-mass
stars to form, that threshold clearly forms only a necessary, not a sufficient,
condition for star formation.  Figure
\ref{fig:coldistributionwithoutstarformation} shows that there is abundant
material within our observed region at $N>1$ g \persc that is not associated
with ongoing high-mass star formation.  These high-column-density, low
star-formation regions are also evident in Figure \ref{fig:coresonscubacol},
which shows that the northern and western regions of the cloud are the
deficient zones.

% nonstarforming_regions.py
\FigureOneCol{figures/column_density_distribution_with_and_without_SF.png}
{Histograms of the column density measured with the combined SCUBA and Herschel
data using the interpolated Herschel temperatures covering only the region
observed with ALMA.  The black histogram shows the whole observed region,
the blue solid shows the SCUBA pixels that do not contain an ALMA source,
and the red dashed region shows those pixels that are within one beam
FWHM of an ALMA source.  While the ALMA sources (high mass protostars)
clearly reside in high-column gas, there is abundant high-column material
that shows no signs of ongoing star formation.}
{fig:coldistributionwithoutstarformation}{1}{0.5\textwidth}

\subsection{What drives star formation in the greater Sgr B2 complex?}
\label{sec:whatdrives}
We have shown that, in addition to the known forming massive clusters,
star formation is ongoing in an extended and elongated region to the north
and south.  Excluding the clusters, most of the newly discovered sources
trace out long and linear features.  Why are the sources aligned?

There is extended ionized emission in Sgr B2 Deep South that appears to be a
bubble surrounded by the millimeter continuum sources.  While this region looks
like a normal, if a bit lumpy, \hii region in the 12\arcsec resolution 20 cm
VLA data in Figure \ref{fig:coreson20cmandhc3n}, the 3 mm continuum reveals long
filamentary features reminiscent of the Galactic center arched filaments.  By
analogy, they may be magnetically dominated regions, but there must be some
central source of ionizing radiation or energetic particles.  Whatever the
driver, it is possible that an expanding bubble of hot gas has compressed
the molecular material along the ridge where we observe star formation.

By contrast, Sgr B2 Far North does not contain any ionized emission.  It
contains fewer total sources, but these sources trace the edge of an expanding
bubble previously noted by \citet{de-Vicente1997a}.  The coincidence of star
forming cores along the edge of a bubble again suggests some sort of
compressional triggering.

However, while both of these regions show circumstantial, morphological evidence
for a compressional event, there are other regions within our map that show
the same general morphology in the gas, yet exhibit no star formation.

The morphological features can be seen in Figure \ref{fig:hc3nonscuba}.  The
molecular gas, as traced by HC$_3$N in this case, outlines bubble edges in all
directions from Sgr B2 M.  The eastern edge shows the clearest bubble in this
image, with the HC$_3$N outlining a cavity in the column density map.  The edge
of this eastern bubble has a lower average column density than either the
northern Sgr B2 NE or the Sgr B2 DS regions, and it shows no signs of ongoing
star formation.  By contrast, the bubble edges in the Sgr B2 NE and the ridge
Sgr B2 W both contain some protostars, and the Sgr B2 DS ridge has the most.

% contour_overlays.py
\FigureOneCol{figures/HC3N_contours_on_SCUBA_column.png}
{ALMA HC$_3$N peak intensity contours (orange) overlaid on the derived SCUBA column density
image using Herschel Hi-Gal interpolated temperatures. The HC$_3$N was
shown in grayscale in Figure \ref{fig:coreson20cmandhc3n}.
Contours are at levels [3,7,11,15,19,23] K.  The  HC$_3$N bubble edges can be
seen surrounding cavities in the SCUBA column density map on the east side of
the main ridge.  To the north, the HC$_3$N also traces bubbles, but these are
less evident in this velocity-integrated view.  The important feature discussed
in Section \ref{sec:whatdrives} is the differing column density around each of
the bubbles.}
{fig:hc3nonscuba}{1}{0.5\textwidth}

\section{Conclusions}
We have reported the detection of \ncores 3 mm point sources in the extended
Sgr B2 cloud and determined that the majority are high-mass protostellar
cores.  This survey represents the first large population of protostars
detected in the Galactic center and represents the largest sample yet reported
of high-mass protostars.


\software{
The software used to make this version of the paper is available from github at
\url{https://github.com/keflavich/SgrB2_ALMA_3mm_Mosaic/} with hash \githash
(\gitdate).}

%\ifstandalone
%\bibliographystyle{apj_w_etal}  % or "siam", or "alpha", or "abbrv"
\bibliographystyle{aasjournal}  % or "siam", or "alpha", or "abbrv"
%\bibliography{thesis}      % bib database file refs.bib
%\bibliography{bibdesk}      % bib database file refs.bib
\bibliography{extracted}      % bib database file refs.bib
%\fi


\appendix

\section{Single Dish Combination}
\label{sec:singledishcomb}
To measure the column density at a resolution similar to \citet{Lada2010a}, we
needed to use ground-based single-dish data with resolution $\sim10\arcsec$.
We combined these images with Herschel data, which recover all angular
scales, to fill in the missing `short spacings' from the ground-based data.

Specifically, we combine the SHARC 350 \um \citep{Dowell1999a} and 
SCUBA 450 \um \citep{Pierce-Price2000a,di-Francesco2008a} with Herschel 350 and
500 \um data \citep{Molinari2016a}, respectively.

Combining single-dish with `interferometer' data, or data that are otherwise
insensitive to large angular scales, is not a trivial process.  The standard
approach advocated by the ALMA project is to use the `feather' process, in
which two images are fourier-transformed, multiplied by a weighting function,
added together, and fourier transformed back to image space \citep[see
equations in \S 5.2 of][]{Stanimirovic2002a}.  This process is subject to
substantial uncertainties, particularly in the choice of the weighting
function.  

Two factors need to be specified for linear combination: the beam size of the
`single-dish', or total power, image, and the largest angular scale of the
`interferometer' or filtered image.  While the beam size is sometimes
well-known, for single dishes operating at the top of their usable frequency
range (e.g., the CSO at 350 \um or GBT at 3 mm), there are uncertainties in the
beam shape and area and there are often substantial sidelobes.  In
interferometric data, the largest angular scale is well-defined in the
originally sampled UV data, but is less well-defined in the final image because
different weighting factors change the recovered largest angular scale.  For
ground-based filtered data, the largest recoverable angular scale is difficult
to determine and requires concerted effort
\citep[e.g.,][]{Ginsburg2013a,Chapin2013a}.

To assess the uncertainties in image combination, particularly on the
brightness distribution \citep[e.g.][]{Ossenkopf-Okada2016a}, we have performed
a series of experiments combining the Herschel with the SCUBA data using
different weights applied to the SCUBA data.  As discussed in Section
\ref{sec:observations}, we empirically determined the scale factor required for
the best match between SCUBA and Herschel data was $3\times$, which is
shockingly large but justifiable.  In the experiment shown in Figure
\ref{fig:feathercompare}, we show the images and resulting histograms when we
combine the Herschel data with the SCUBA data scaled by a range of factors from
$0.5\times$ to $10\times$.  The changes to the high end of the histogram are
dramatic, but the middle region containing most of the pixels (and most
relevant to the discussion of thresholds in the paper) is hardly affected.
Additionally, we show the cumulative distribution function of core background
surface brightnesses (as in Figure \ref{fig:corebackgroundcdf}), showing again
that only the high end is affected. 

\Figure{figures/scuba_feather_scalefactor_comparison.pdf}
{A demonstration of the effects of using different calibration factors when
combining the SCUBA data with the Herschel data using the `feather' process.
The numbers above each panel show the scale factor applied to the SCUBA data
before fourier-combining it with the Herschel data.  The factor of 3 was used
in this paper and shows the most reasonable balance between the high-resolution
of the SCUBA data and the all-positive Herschel data.  In the lower panels, the
fiducial scale factor of 3 is shown in black in all panels.  The solid lines
show histograms of the images displayed in the top panels.  The dashed lines
show the cumulative distribution of the background surface brightnesses of the
point sources in this sample; they are similar to the distributions shown in
Figure \ref{fig:corebackgroundcdf}.}
{fig:feathercompare}{1}{\textwidth}

\section{Self-calibration}
\label{sec:selfcal}
We demonstrate the impact of self-calibration in this section.  The adopted approach
used three iterations of phase-only self-calibration followed by two iterations of
phase and amplitude self-calibration.  Each iteration involved slightly
different imaging parameters.  The final, deepest clean used a threshold mask
on the previous shallower clean. The script used to produce the final images is
available at
\url{https://github.com/keflavich/SgrB2_ALMA_3mm_Mosaic/blob/\githash/script_merge/selfcal_continuum_merge_7m.py}.
The effects are shown with a cutout centered on the most affected region around
Sgr B2 M in Figure \ref{fig:selfcalprogression}.

% selfcal_progression
\Figure{figures/selfcal_progression_TCTE7m_SgrB2M.pdf}
{Progression of the self-calibration iterations.  The images show, from left to
right, the initial image, one, two, and three iterations of phase-only self
calibration, one iteration of phase and amplitude self-calibration, one
iteration of phase and amplitude self-calibration using two Taylor terms and
multiscale clean for the imaging, and finally, a reimaging of the last
iteration with a deeper 0.1 mJy threshold using a mask at the 2.5 mJy level.
The second row shows the corresponding residual images.}
{fig:selfcalprogression}{1}{\textwidth}

\section{Photometric Catalog}
We include the full catalog here.

\begin{table}[htp]
\caption{Continuum Source IDs and photometry}
\begin{tabular}{lllllllllll}
\label{tab:photometry}
\footnotesize
ID & Coordinates & $S_{nu,max}$ & $T_{B,max}$ & $S_{nu,tot}$ & $\sigma_{bg}$ & $\alpha$ & $E(\alpha)$ & $M_{20K}$ & $N(\hh)_{20 K}$ & Classification \\
 &  & mJy bm$^{-1}$ & $\mathrm{K}$ & $\mathrm{mJy}$ & mJy bm$^{-1}$ &  &  & $\mathrm{M_{\odot}}$ & $\mathrm{cm^{-2}}$ &  \\
\hline
174 f3 & 17:47:20.167 -28:23:04.809 & 16 & 86 & 79 & 46 & 0.89 &  & 59 & - & SX\_ HII \\
234 f4 & 17:47:20.214 -28:23:04.379 & 11 & 57 & 3 & 23 & 0.83 &  & 39 & - & SX\_ HII \\
176 f1 & 17:47:20.127 -28:23:04.082 & 92 & 48 & 48 & 3 & 1.2 & 0.01 & 33 & - & SX\_ denseCore \\
236 f10.303 & 17:47:20.106 -28:23:03.729 & 89 & 46 & 26 & 19 & 1.1 & 0.01 & 32 & - & S\_\_ HII \\
235 f2 & 17:47:20.166 -28:23:03.714 & 82 & 43 & 22 & 33 & 1.3 &  & 29 & - & S\_\_ HII \\
172 K2 & 17:47:19.869 -28:22:18.466 & 37 & 2 & 22 & 49 & 2.5 & 0.02 & 13 & - & S\_\_ denseCore \\
175 G & 17:47:20.285 -28:23:03.162 & 34 & 18 & 13 & 5.6 & 0.68 & 0.03 & 12 & - & S\_\_ HII \\
237 G10.44 & 17:47:20.241 -28:23:03.387 & 28 & 14 & 52 & 15 & 0.69 & 0.01 & 98 & - & S\_\_ HII \\
178 f10.37 & 17:47:20.178 -28:23:06 & 2 & 1 & 89 & 18 & 1.5 & 0.04 & 71 & - & SX\_ HII \\
171 K3 & 17:47:19.895 -28:22:17.221 & 19 & 97 & 94 & 25 & 1.4 & 0.02 & 66 & - & S\_\_ HII \\
177 B & 17:47:19.918 -28:23:03.039 & 15 & 77 & 79 & 3.9 & 0.47 & 0.01 & 53 & - & S\_M HII \\
241 & 17:47:20.106 -28:23:03.066 & 14 & 73 & 4 & 15 & 1.4 & 0.05 & 5 & - & S\_\_ denseCore \\
179 f10.38 & 17:47:20.193 -28:23:06.673 & 13 & 66 & 59 & 9.3 & 1.6 & 0.01 & 45 & - & SX\_ HII \\
180 E & 17:47:20.108 -28:23:08.894 & 13 & 66 & 63 & 4 & 0.38 & 0.01 & 45 & - & S\_\_ HII \\
173 K1 & 17:47:19.78 -28:22:20.743 & 92 & 48 & 49 & 4.4 & 0.58 & 0.03 & 33 & - & S\_\_ HII \\
170 & 17:47:19.895 -28:22:13.621 & 92 & 48 & 52 & 23 & 1.7 & 0.08 & 33 & - & S\_\_ PartofCloud \\
225 f10.33b & 17:47:20.116 -28:23:06.374 & 69 & 36 & 34 & 14 & 1.9 & 0.21 & 25 & - & SX\_ denseCore \\
96 Z10.24 & 17:47:20.039 -28:22:41.25 & 64 & 33 & 25 & 1.5 & 0.68 & 0.37 & 23 & - & S\_M Maser \\
181 D & 17:47:20.051 -28:23:12.91 & 59 & 31 & 31 & 1.3 & 0.64 & 0.09 & 21 & - & S\_M HII \\
240 f10.44b & 17:47:20.252 -28:23:06.463 & 57 & 3 & 17 & 11 & 1.8 & 0.01 & 2 & - & SX\_ HII \\
233 f10.27b & 17:47:20.077 -28:23:05.383 & 5 & 26 & 26 & 18 & 2.3 & 0.18 & 18 & - & SX\_ HII \\
239 & 17:47:20.242 -28:23:07.222 & 45 & 24 & 15 & 8.6 & 2.3 & 0.09 & 16 & - & SX\_ denseCore \\
242 & 17:47:20.129 -28:23:02.247 & 32 & 17 & 21 & 8.5 & 2.2 & 0.1 & 11 & 3.3\ee{26} & S\_\_ denseCore \\
92 I10.52 & 17:47:20.324 -28:23:08.2 & 32 & 17 & 15 & 5.3 & 0.63 & 0.06 & 11 & 3.1\ee{26} & S\_\_ HII \\
109 & 17:47:19.901 -28:22:15.54 & 24 & 13 & 14 & 13 & 3.6 & 0.3 & 86 & 1.4\ee{26} & S\_\_ - \\
87 B9.99 & 17:47:19.798 -28:23:06.942 & 23 & 12 & 12 & 1.9 & 0.89 & 0.04 & 83 & 1.3\ee{26} & S\_\_ HII \\
88 & 17:47:19.617 -28:23:08.26 & 23 & 12 & 11 & 2.9 & 3.1 & 0.18 & 81 & 1.3\ee{26} & S\_\_ - \\
151 B10.06 & 17:47:19.86 -28:23:01.5 & 21 & 11 & 1 & 1.3 & 0.19 & 0.79 & 73 & 1.1\ee{26} & S\_M HII \\
98 & 17:47:19.53 -28:22:32.55 & 18 & 9.5 & 95 & 0.36 & 3.2 & 1.1 & 64 & 8.7\ee{25} & S\_M Maser \\
152 f10.32 & 17:47:20.128 -28:23:00.22 & 16 & 8.5 & 89 & 3 & -0.3 & 0.26 & 58 & 7.5\ee{25} & S\_\_ HII \\
86 B9.96 & 17:47:19.766 -28:23:10.183 & 16 & 8.3 & 81 & 1.8 & 1 & 0.09 & 57 & 7.2\ee{25} & S\_\_ HII \\
76 & 17:47:19.986 -28:23:48.86 & 15 & 7.9 & 7 & 0.94 & 3.3 & 0.35 & 54 & 6.8\ee{25} & S\_\_ - \\
97 & 17:47:19.838 -28:22:40.07 & 13 & 6.8 & 67 & 1.1 & 3 & 0.57 & 46 & 5.5\ee{25} & S\_\_ - \\
85 & 17:47:18.029 -28:23:03.61 & 13 & 6.7 & 47 & 0.17 & 3 & 0.32 & 46 & 5.4\ee{25} & S\_\_ - \\
\hline
\end{tabular}
\par
The Classification column consists of three letter codes as described in Section \ref{sec:classification}.  In column 1, \texttt{S} indicates a strong source, \texttt{W} indicates weak or low-confidence source. In column 2, an \texttt{X} indicates a match with the \citet{Muno2009a} Chandra X-ray source catalog, while anunderscore indicates there was no match.In column 3, \texttt{M} indicates a match with the, \citet{Caswell2010a} Methanol Multibeam Survey \methanol maser catalog, while an underscore indicates there was no match.  Finally, we include the SIMBAD \citep{Wenger2000a} source object type classification if one was found.
\end{table}


\end{document}

\documentclass{emulateapj}
%\documentclass[defaultstyle,11pt]{thesis}
%\documentclass[]{report}
%\documentclass[]{article}
%\usepackage{aastex_hack}
%\usepackage{deluxetable}
%\documentclass[preprint]{aastex}
%\documentclass{aa}

\newcommand{\titlerunning}[1]{\shorttitle{#1}}
\newcommand{\authorrunning}[1]{\shortauthors{#1}}

\newcommand*\inst[1]{\unskip\hbox{\@textsuperscript{\normalfont$#1$}}}

%\newcount\aa@nbinstitutes
%
%\newcounter{aa@institutecnt}

\newcommand*\institute[1]{
  \begingroup
    \let\and\relax
    \renewcommand*\inst[1]{}%
    \renewcommand*\thanks[1]{}%
    \renewcommand*\email[1]{}%
    %\let\@@protect\protect
    %\let\protect\@unexpandable@protect
    %\global\aa@nbinstitutes \z@
    %\expandafter\aa@cntinstitutes\aa@institute\and\aa@nil\and
    %\restore@protect
  \endgroup
  \newcommand{\institutions}{#1}
}%


%\renewcommand{\abstract}[1]{
%\begin{abstract}
%    #1
%\end{abstract}
%}

%\renewcommand\ion[2]{#1$\;${%
%\ifx\@currsize\normalsize\small \else
%\ifx\@currsize\small\footnotesize \else
%\ifx\@currsize\footnotesize\scriptsize \else
%\ifx\@currsize\scriptsize\tiny \else
%\ifx\@currsize\large\normalsize \else
%\ifx\@currsize\Large\large
%\fi\fi\fi\fi\fi\fi
%\rmfamily\@Roman{#2}}\relax}% 
%
%\renewcommand{ion}[2]{#1}{#2}

\renewcommand{\ion}[2]{\textup{#1\,\textsc{\lowercase{#2}}}}

%\newcommand{\uchii}{\ensuremath{\mathrm{\ion{UCH}{2}}}\xspace}
%\newcommand{\UCHII}{\ensuremath{\mathrm{\ion{UCH}{2}}}\xspace}
%\newcommand{\hchii}{\ensuremath{\mathrm{\ion{HCH}{2}}}\xspace}
%\newcommand{\HCHII}{\ensuremath{\mathrm{\ion{HCH}{2}}}\xspace}
%\newcommand{\hii}  {\ensuremath{\mathrm{\ion{H}{2}}}\xspace}

%\input{aamacros.tex}

\pdfminorversion=4


%%%%%%%%%%%%%%%%%%%%%%%%%%%%%%%%%%%%%%%%%%%%%%%%%%%%%%%%%%%%%%%%
%%%%%%%%%%%  see documentation for information about  %%%%%%%%%%
%%%%%%%%%%%  the options (11pt, defaultstyle, etc.)   %%%%%%%%%%
%%%%%%%  http://www.colorado.edu/its/docs/latex/thesis/  %%%%%%%
%%%%%%%%%%%%%%%%%%%%%%%%%%%%%%%%%%%%%%%%%%%%%%%%%%%%%%%%%%%%%%%%
%		\documentclass[typewriterstyle]{thesis}
% 		\documentclass[modernstyle]{thesis}
% 		\documentclass[modernstyle,11pt]{thesis}
%	 	\documentclass[modernstyle,12pt]{thesis}

%%%%%%%%%%%%%%%%%%%%%%%%%%%%%%%%%%%%%%%%%%%%%%%%%%%%%%%%%%%%%%%%
%%%%%%%%%%%    load any packages which are needed    %%%%%%%%%%%
%%%%%%%%%%%%%%%%%%%%%%%%%%%%%%%%%%%%%%%%%%%%%%%%%%%%%%%%%%%%%%%%
\usepackage{latexsym}		% to get LASY symbols
\usepackage{graphicx}		% to insert PostScript figures
%\usepackage{deluxetable}
\usepackage{rotating}		% for sideways tables/figures
\usepackage{natbib}  % Requires natbib.sty, available from http://ads.harvard.edu/pubs/bibtex/astronat/
\usepackage{savesym}
%\usepackage{pdflscape}
\usepackage{amssymb}
\usepackage{morefloats}
%\savesymbol{singlespace}
\savesymbol{doublespace}
%\usepackage{wrapfig}
%\usepackage{setspace}
\usepackage{xspace}
\usepackage{color}
%\usepackage{multicol}
\usepackage{mdframed}
\usepackage{url}
\usepackage{subfigure}
%\usepackage{emulateapj}
%\usepackage{lscape}
\usepackage{grffile}
\usepackage{standalone}
\standalonetrue
\usepackage{import}
\usepackage[utf8]{inputenc}
\usepackage{longtable}
\usepackage{booktabs}
\usepackage[yyyymmdd,hhmmss]{datetime}
\usepackage{fancyhdr}
\usepackage[colorlinks=true,citecolor=blue,linkcolor=cyan]{hyperref}
\usepackage{ifpdf}







\newcommand{\paa}{Pa\ensuremath{\alpha}}
\newcommand{\brg}{Br\ensuremath{\gamma}}
\newcommand{\msun}{\ensuremath{M_{\odot}}\xspace}			%  Msun
\newcommand{\mdot}{\ensuremath{\dot{M}}\xspace}
\newcommand{\lsun}{\ensuremath{L_{\odot}}\xspace}			%  Lsun
\newcommand{\rsun}{\ensuremath{R_{\odot}}\xspace}			%  Rsun
\newcommand{\lbol}{\ensuremath{L_{\mathrm{bol}}\xspace}}	%  Lbol
\newcommand{\ks}{K\ensuremath{_{\mathrm{s}}}}		%  Ks
\newcommand{\hh}{\ensuremath{\textrm{H}_{2}}\xspace}			%  H2
\newcommand{\dens}{\ensuremath{n(\hh) [\percc]}\xspace}
\newcommand{\formaldehyde}{\ensuremath{\textrm{H}_2\textrm{CO}}\xspace}
\newcommand{\formamide}{\ensuremath{\textrm{NH}_2\textrm{CHO}}\xspace}
\newcommand{\formaldehydeIso}{\ensuremath{\textrm{H}_2~^{13}\textrm{CO}}\xspace}
\newcommand{\methanol}{\ensuremath{\textrm{CH}_3\textrm{OH}}\xspace}
\newcommand{\ortho}{\ensuremath{\textrm{o-H}_2\textrm{CO}}\xspace}
\newcommand{\para}{\ensuremath{\textrm{p-H}_2\textrm{CO}}\xspace}
\newcommand{\oneone}{\ensuremath{1_{1,0}-1_{1,1}}\xspace}
\newcommand{\twotwo}{\ensuremath{2_{1,1}-2_{1,2}}\xspace}
\newcommand{\threethree}{\ensuremath{3_{1,2}-3_{1,3}}\xspace}
\newcommand{\threeohthree}{\ensuremath{3_{0,3}-2_{0,2}}\xspace}
\newcommand{\threetwotwo}{\ensuremath{3_{2,2}-2_{2,1}}\xspace}
\newcommand{\threetwoone}{\ensuremath{3_{2,1}-2_{2,0}}\xspace}
\newcommand{\fourtwotwo}{\ensuremath{4_{2,2}-3_{1,2}}\xspace} % CH3OH 218.4 GHz
\newcommand{\methylcyanide}{\ensuremath{\textrm{CH}_{3}\textrm{CN}}\xspace}
\newcommand{\ketene}{\ensuremath{\textrm{H}_{2}\textrm{CCO}}\xspace}
\newcommand{\ethylcyanide}{\ensuremath{\textrm{CH}_3\textrm{CH}_2\textrm{CN}}\xspace}
\newcommand{\cyanoacetylene}{\ensuremath{\textrm{HC}_{3}\textrm{N}}\xspace}
\newcommand{\methylformate}{\ensuremath{\textrm{CH}_{3}\textrm{OCHO}}\xspace}
\newcommand{\dimethylether}{\ensuremath{\textrm{CH}_{3}\textrm{OCH}_{3}}\xspace}
\newcommand{\gaucheethanol}{\ensuremath{\textrm{g-CH}_3\textrm{CH}_2\textrm{OH}}\xspace}
\newcommand{\acetone}{\ensuremath{\left[\textrm{CH}_{3}\right]_2\textrm{CO}}\xspace}
\newcommand{\methyleneamidogen}{\ensuremath{\textrm{H}_{2}\textrm{CN}}\xspace}
\newcommand{\Rone}{\ensuremath{\para~S_{\nu}(\threetwoone) / S_{\nu}(\threeohthree)}\xspace}
\newcommand{\Rtwo}{\ensuremath{\para~S_{\nu}(\threetwotwo) / S_{\nu}(\threetwoone)}\xspace}
\newcommand{\JKaKc}{\ensuremath{J_{K_a K_c}}}
\newcommand{\water}{H$_{2}$O\xspace}		%  H2O
\newcommand{\feii}{\ion{Fe}{ii}\xspace}		%  FeII

\newcommand{\uchii}{\ion{UCH}{ii}\xspace}
\newcommand{\UCHII}{\ion{UCH}{ii}\xspace}
\newcommand{\hchii}{\ion{HCH}{ii}\xspace}
\newcommand{\HCHII}{\ion{HCH}{ii}\xspace}
\newcommand{\hii}{\ion{H}{ii}\xspace}

\newcommand{\hi}{H~{\sc i}\xspace}
\newcommand{\Hii}{\hii}
\newcommand{\HII}{\hii}
\newcommand{\Xform}{\ensuremath{X_{\formaldehyde}}}
\newcommand{\kms}{\textrm{km~s}\ensuremath{^{-1}}\xspace}	%  km s-1
\newcommand{\nsample}{456\xspace}
\newcommand{\CFR}{5\xspace} % nMPC / 0.25 / 2 (6 for W51 once, 8 for W51 twice) REFEDIT: With f_observed=0.3, becomes 3/2./0.3 = 5
\newcommand{\permyr}{\ensuremath{\mathrm{Myr}^{-1}}\xspace}
\newcommand{\pers}{\ensuremath{\mathrm{s}^{-1}}\xspace}
\newcommand{\tsuplim}{0.5\xspace} % upper limit on starless timescale
\newcommand{\ncandidates}{18\xspace}
\newcommand{\mindist}{8.7\xspace}
\newcommand{\rcluster}{2.5\xspace}
\newcommand{\ncomplete}{13\xspace}
\newcommand{\middistcut}{13.0\xspace}
\newcommand{\nMPC}{3\xspace} % only count W51 once.  W51, W49, G010
\newcommand{\obsfrac}{30}
\newcommand{\nMPCtot}{10\xspace} % = nmpc / obsfrac
\newcommand{\nMPCtoterr}{6\xspace} % = sqrt(nmpc) / obsfrac
\newcommand{\plaw}{2.1\xspace}
\newcommand{\plawerr}{0.3\xspace}
\newcommand{\mmin}{\ensuremath{10^4~\msun}\xspace}
%\newcommand{\perkmspc}{\textrm{per~km~s}\ensuremath{^{-1}}\textrm{pc}\ensuremath{^{-1}}\xspace}	%  km s-1 pc-1
\newcommand{\kmspc}{\textrm{km~s}\ensuremath{^{-1}}\textrm{pc}\ensuremath{^{-1}}\xspace}	%  km s-1 pc-1
\newcommand{\sqcm}{cm$^{2}$\xspace}		%  cm^2
\newcommand{\percc}{\ensuremath{\textrm{cm}^{-3}}\xspace}
\newcommand{\perpc}{\ensuremath{\textrm{pc}^{-1}}\xspace}
\newcommand{\persc}{\ensuremath{\textrm{cm}^{-2}}\xspace}
\newcommand{\persr}{\ensuremath{\textrm{sr}^{-1}}\xspace}
\newcommand{\peryr}{\ensuremath{\textrm{yr}^{-1}}\xspace}
\newcommand{\perkmspc}{\textrm{km~s}\ensuremath{^{-1}}\textrm{pc}\ensuremath{^{-1}}\xspace}	%  km s-1 pc-1
\newcommand{\perkms}{\textrm{per~km~s}\ensuremath{^{-1}}\xspace}	%  km s-1 
\newcommand{\um}{\ensuremath{\mu \textrm{m}}\xspace}    % micron
\newcommand{\microjy}{\ensuremath{\mu\textrm{Jy}}\xspace}    % micron
\newcommand{\mum}{\um}
\newcommand{\htwo}{\ensuremath{\textrm{H}_2}}
\newcommand{\Htwo}{\ensuremath{\textrm{H}_2}}
\newcommand{\HtwoO}{\ensuremath{\textrm{H}_2\textrm{O}}}
\newcommand{\htwoo}{\ensuremath{\textrm{H}_2\textrm{O}}}
\newcommand{\ha}{\ensuremath{\textrm{H}\alpha}}
\newcommand{\hb}{\ensuremath{\textrm{H}\beta}}
\newcommand{\so}{SO~\ensuremath{5_6-4_5}\xspace}
\newcommand{\SO}{SO~\ensuremath{1_2-1_1}\xspace}
\newcommand{\ammonia}{NH\ensuremath{_3}\xspace}
\newcommand{\twelveco}{\ensuremath{^{12}\textrm{CO}}\xspace}
\newcommand{\thirteenco}{\ensuremath{^{13}\textrm{CO}}\xspace}
\newcommand{\ceighteeno}{\ensuremath{\textrm{C}^{18}\textrm{O}}\xspace}
\def\ee#1{\ensuremath{\times10^{#1}}}
\newcommand{\degrees}{\ensuremath{^{\circ}}}
% can't have \degree because I'm getting a degree...
\newcommand{\lowirac}{800}
\newcommand{\highirac}{8000}
\newcommand{\lowmips}{600}
\newcommand{\highmips}{5000}
\newcommand{\perbeam}{\ensuremath{\textrm{beam}^{-1}}}
\newcommand{\ds}{\ensuremath{\textrm{d}s}}
\newcommand{\dnu}{\ensuremath{\textrm{d}\nu}}
\newcommand{\dv}{\ensuremath{\textrm{d}v}}
\def\secref#1{Section \ref{#1}}
\def\eqref#1{Equation \ref{#1}}
\def\facility#1{#1}
%\newcommand{\arcmin}{'}

\newcommand{\necluster}{Sh~2-233IR~NE}
\newcommand{\swcluster}{Sh~2-233IR~SW}
\newcommand{\region}{IRAS 05358}

\newcommand{\nwfive}{40}
\newcommand{\nouter}{15}

\newcommand{\vone}{{\rm v}1.0\xspace}
\newcommand{\vtwo}{{\rm v}2.0\xspace}
\newcommand\mjysr{\ensuremath{{\rm MJy~sr}^{-1}}}
\newcommand\jybm{\ensuremath{{\rm Jy~bm}^{-1}}}
\newcommand\nbolocat{8552\xspace}
\newcommand\nbolocatnew{548\xspace}
\newcommand\nbolocatnonew{8004\xspace} % = nbolocat-nbolocatnew
%\renewcommand\arcdeg{\mbox{$^\circ$}\xspace} 
%\renewcommand\arcmin{\mbox{$^\prime$}\xspace} 
%\renewcommand\arcsec{\mbox{$^{\prime\prime}$}\xspace} 

\newcommand{\todo}[1]{\textcolor{red}{#1}}
\newcommand{\okinfinal}[1]{{#1}}
%% only needed if not aastex
%\newcommand{\keywords}[1]{}
%\newcommand{\email}[1]{}
%\newcommand{\affil}[1]{}


%aastex hack
%\newcommand\arcdeg{\mbox{$^\circ$}}%
%\newcommand\arcmin{\mbox{$^\prime$}\xspace}%
%\newcommand\arcsec{\mbox{$^{\prime\prime}$}\xspace}%

%\newcommand\epsscale[1]{\gdef\eps@scaling{#1}}
%
%\newcommand\plotone[1]{%
% \typeout{Plotone included the file #1}
% \centering
% \leavevmode
% \includegraphics[width={\eps@scaling\columnwidth}]{#1}%
%}%
%\newcommand\plottwo[2]{{%
% \typeout{Plottwo included the files #1 #2}
% \centering
% \leavevmode
% \columnwidth=.45\columnwidth
% \includegraphics[width={\eps@scaling\columnwidth}]{#1}%
% \hfil
% \includegraphics[width={\eps@scaling\columnwidth}]{#2}%
%}}%


%\newcommand\farcm{\mbox{$.\mkern-4mu^\prime$}}%
%\let\farcm\farcm
%\newcommand\farcs{\mbox{$.\!\!^{\prime\prime}$}}%
%\let\farcs\farcs
%\newcommand\fp{\mbox{$.\!\!^{\scriptscriptstyle\mathrm p}$}}%
%\newcommand\micron{\mbox{$\mu$m}}%
%\def\farcm{%
% \mbox{.\kern -0.7ex\raisebox{.9ex}{\scriptsize$\prime$}}%
%}%
%\def\farcs{%
% \mbox{%
%  \kern  0.13ex.%
%  \kern -0.95ex\raisebox{.9ex}{\scriptsize$\prime\prime$}%
%  \kern -0.1ex%
% }%
%}%

\def\Figure#1#2#3#4#5{
\begin{figure*}[!htp]
\includegraphics[scale=#4,width=#5]{#1}
\caption{#2}
\label{#3}
\end{figure*}
}

\def\WrapFigure#1#2#3#4#5#6{
\begin{wrapfigure}{#6}{0.5\textwidth}
\includegraphics[scale=#4,width=#5]{#1}
\caption{#2}
\label{#3}
\end{wrapfigure}
}

% % #1 - filename
% % #2 - caption
% % #3 - label
% % #4 - epsscale
% % #5 - R or L?
% \def\WrapFigure#1#2#3#4#5#6{
% \begin{wrapfigure}[#6]{#5}{0.45\textwidth}
% %  \centercaption
% %  \vspace{-14pt}
%   \epsscale{#4}
%   \includegraphics[scale=#4]{#1}
%   \caption{#2}
%   \label{#3}
% \end{wrapfigure}
% }

\def\RotFigure#1#2#3#4#5{
\begin{sidewaysfigure*}[!htp]
\includegraphics[scale=#4,width=#5]{#1}
\caption{#2}
\label{#3}
\end{sidewaysfigure*}
}

\def\FigureSVG#1#2#3#4{
\begin{figure*}[!htp]
    \def\svgwidth{#4}
    \input{#1}
    \caption{#2}
    \label{#3}
\end{figure*}
}

% originally intended to be included in a two-column paper
% this is in includegraphics: ,width=3in
% but, not for thesis
\def\OneColFigure#1#2#3#4#5{
\begin{figure}[!htpb]
\epsscale{#4}
\includegraphics[scale=#4,angle=#5]{#1}
\caption{#2}
\label{#3}
\end{figure}
}

\def\SubFigure#1#2#3#4#5{
\begin{figure*}[!htp]
\addtocounter{figure}{-1}
\epsscale{#4}
\includegraphics[angle=#5]{#1}
\caption{#2}
\label{#3}
\end{figure*}
}

%\def\FigureTwo#1#2#3#4#5{
%\begin{figure*}[!htp]
%\epsscale{#5}
%\plottwo{#1}{#2}
%\caption{#3}
%\label{#4}
%\end{figure*}
%}

\def\FigureTwo#1#2#3#4#5#6{
\begin{figure*}[!htp]
\subfigure[]{ \includegraphics[scale=#5,width=#6]{#1} }
\subfigure[]{ \includegraphics[scale=#5,width=#6]{#2} }
\caption{#3}
\label{#4}
\end{figure*}
}

\def\FigureTwoAA#1#2#3#4#5#6{
\begin{figure*}[!htp]
\subfigure[]{ \includegraphics[scale=#5,width=#6]{#1} }
\subfigure[]{ \includegraphics[scale=#5,width=#6]{#2} }
\caption{#3}
\label{#4}
\end{figure*}
}

\newenvironment{rotatepage}%
{}{}
   %{\pagebreak[4]\afterpage\global\pdfpageattr\expandafter{\the\pdfpageattr/Rotate 90}}%
   %{\pagebreak[4]\afterpage\global\pdfpageattr\expandafter{\the\pdfpageattr/Rotate 0}}%


\def\RotFigureTwoAA#1#2#3#4#5#6{
\begin{rotatepage}
\begin{sidewaysfigure*}[!htp]
\subfigure[]{ \includegraphics[scale=#5,width=#6]{#1} }
\\
\subfigure[]{ \includegraphics[scale=#5,width=#6]{#2} }
\caption{#3}
\label{#4}
\end{sidewaysfigure*}
\end{rotatepage}
}

\def\RotFigureThreeAA#1#2#3#4#5#6#7{
\begin{rotatepage}
\begin{sidewaysfigure*}[!htp]
\subfigure[]{ \includegraphics[scale=#6,width=#7]{#1} }
\\
\subfigure[]{ \includegraphics[scale=#6,width=#7]{#2} }
\\
\subfigure[]{ \includegraphics[scale=#6,width=#7]{#3} }
\caption{#4}
\label{#5}
\end{sidewaysfigure*}
\end{rotatepage}
\clearpage
}

\def\FigureThreeAA#1#2#3#4#5#6#7{
\begin{figure*}[!htp]
\subfigure[]{ \includegraphics[scale=#6,width=#7]{#1} }
\subfigure[]{ \includegraphics[scale=#6,width=#7]{#2} }
\subfigure[]{ \includegraphics[scale=#6,width=#7]{#3} }
\caption{#4}
\label{#5}
\end{figure*}
}



\def\SubFigureTwo#1#2#3#4#5{
\begin{figure*}[!htp]
\addtocounter{figure}{-1}
\epsscale{#5}
\plottwo{#1}{#2}
\caption{#3}
\label{#4}
\end{figure*}
}

\def\FigureFour#1#2#3#4#5#6{
\begin{figure*}[!htp]
\subfigure[]{ \includegraphics[width=3in]{#1} }
\subfigure[]{ \includegraphics[width=3in]{#2} }
\subfigure[]{ \includegraphics[width=3in]{#3} }
\subfigure[]{ \includegraphics[width=3in]{#4} }
\caption{#5}
\label{#6}
\end{figure*}
}

\def\FigureFourPDF#1#2#3#4#5#6{
\begin{figure*}[!htp]
\subfigure[]{ \includegraphics[width=3in,type=pdf,ext=.pdf,read=.pdf]{#1} }
\subfigure[]{ \includegraphics[width=3in,type=pdf,ext=.pdf,read=.pdf]{#2} }
\subfigure[]{ \includegraphics[width=3in,type=pdf,ext=.pdf,read=.pdf]{#3} }
\subfigure[]{ \includegraphics[width=3in,type=pdf,ext=.pdf,read=.pdf]{#4} }
\caption{#5}
\label{#6}
\end{figure*}
}

\def\FigureThreePDF#1#2#3#4#5{
\begin{figure*}[!htp]
\subfigure[]{ \includegraphics[width=3in,type=pdf,ext=.pdf,read=.pdf]{#1} }
\subfigure[]{ \includegraphics[width=3in,type=pdf,ext=.pdf,read=.pdf]{#2} }
\subfigure[]{ \includegraphics[width=3in,type=pdf,ext=.pdf,read=.pdf]{#3} }
\caption{#4}
\label{#5}
\end{figure*}
}

\def\Table#1#2#3#4#5{
%\renewcommand{\thefootnote}{\alph{footnote}}
\begin{table}
\caption{#2}
\label{#3}
    \begin{tabular}{#1}
        \hline\hline
        #4
        \hline
        #5
        \hline
    \end{tabular}
\end{table}
%\renewcommand{\thefootnote}{\arabic{footnote}}
}


%\def\Table#1#2#3#4#5#6{
%%\renewcommand{\thefootnote}{\alph{footnote}}
%\begin{deluxetable}{#1}
%\tablewidth{0pt}
%\tabletypesize{\footnotesize}
%\tablecaption{#2}
%\tablehead{#3}
%\startdata
%\label{#4}
%#5
%\enddata
%\bigskip
%#6
%\end{deluxetable}
%%\renewcommand{\thefootnote}{\arabic{footnote}}
%}

%\def\tablenotetext#1#2{
%\footnotetext[#1]{#2}
%}

% \def\LongTable#1#2#3#4#5#6#7#8{
% % required to get tablenotemark to work: http://www2.astro.psu.edu/users/stark/research/psuthesis/longtable.html
% \renewcommand{\thefootnote}{\alph{footnote}}
% \begin{longtable}{#1}
% \caption[#2]{#2}
% \label{#4} \\
% 
%  \\
% \hline 
% #3 \\
% \hline
% \endfirsthead
% 
% \hline
% #3 \\
% \hline
% \endhead
% 
% \hline
% \multicolumn{#8}{r}{{Continued on next page}} \\
% \hline
% \endfoot
% 
% \hline 
% \endlastfoot
% #7 \\
% 
% #5
% \hline
% #6 \\
% 
% \end{longtable}
% \renewcommand{\thefootnote}{\arabic{footnote}}
% }

\def\TallFigureTwo#1#2#3#4#5#6{
\begin{figure*}[htp]
\epsscale{#5}
\subfigure[]{ \includegraphics[width=#6]{#1} }
\subfigure[]{ \includegraphics[width=#6]{#2} }
\caption{#3}
\label{#4}
\end{figure*}
}

		% file containing author's macro definitions

\newcommand{\ncores}{138\xspace}

\begin{document}
\title{A catalog of 3mm point sources in the Sgr B2 cloud: signs of extended star formation in a CMZ cloud}
\titlerunning{Sgr B2 ALMA}
\authorrunning{Ginsburg et al}
% for future reference, this is probably a better approach:
% https://github.com/dfm/peerless/blob/af483ced97045c213650ed807c430b2f87d2c8c9/document/ms.tex#L104
% assuming it's compatible with A&A
%\newcommand{\nrao}{$^{1}$}
%\newcommand{\eso}{$^{2}$}
\newcommand{\nraojansky}{\affiliation{\it{Jansky fellow of the National Radio Astronomy Observatory, 1003 Lopezville Rd, Socorro, NM 87801 USA }}}
\newcommand{\nrao}{\affiliation{\it{National Radio Astronomy Observatory, 1003 Lopezville Rd, Socorro, NM 87801 USA }}}
\newcommand{\nraocv}{\affiliation{\it{National Radio Astronomy Observatory, 520 Edgemont Rd, Charlottesville, VA 22903, USA }}}
\newcommand{\eso}{ \affiliation{\it{ European Southern Observatory, Karl-Schwarzschild-Stra{\ss}e 2, D-85748 Garching bei München, Germany } } }


\newcommand{\radboud}{\affiliation{\it{Department of Astrophysics/IMAPP, Radboud University Nijmegen, PO Box 9010, 6500 GL Nijmegen, the Netherlands}}}
\newcommand{\allegro}{\affiliation{\it{ALLEGRO/Leiden Observatory, Leiden University, PO Box 9513, 2300 RA Leiden, the Netherlands}}}
\newcommand{\zah}{\affiliation{\it{Astronomisches Rechen-Institut, Zentrum f{\"u}r Astronomie der Universit{\"a}t Heidelberg, M{\"o}nchhofstra{\ss}e 12-14, 69120 Heidelberg, Germany}}}
\newcommand{\casa}{\affiliation{\it{CASA, University of Colorado, 389-UCB, Boulder, CO 80309}} }
\newcommand{\jodrell}{\affiliation{\it{Jodrell Bank Centre for Astrophysics, School of Physics and Astronomy, University of Manchester, Oxford Road, Manchester M13 9PL, UK}}}
\newcommand{\morelia}{\affiliation{\it{Instituto de Radioastronom{\'i}a y Astrof{\'i}sica, UNAM, A.P. 3-72, Xangari, Morelia, 58089, Mexico}}}
\newcommand{\sjsu}{\affiliation{\it{{San Jose State University, One Washington Square, San Jose, CA 95192}}}}
\newcommand{\herts}{\affiliation{\it{Centre for Astrophysics Research, University of Hertfordshire, College Lane, Hatfield, AL10 9AB, UK}}}
\newcommand{\uofa}{\affiliation{\it{Dept. of Physics, University of Alberta, Edmonton, Alberta, Canada}}}
\newcommand{\arcetri}{\affiliation{\it{INAF-Osservatorio Astrofisico di Arcetri, Largo E. Fermi 5, I-50125, Florence, Italy } } }
\newcommand{\exclus}{\affiliation{\it{Excellence Cluster Universe, Boltzman str. 2, D-85748 Garching bei M\"unchen, Germany } }}
\newcommand{\ljmu}{\affiliation{\it{Astrophysics Research Institute, Liverpool John Moores University, 146 Brownlow Hill, Liverpool L3 5RF, UK }}}
\newcommand{\koeln}{\affiliation{\it{I. Physikalisches Institut, Universi\"at zu K\"oln, Z\"ulpicher Str.\ 77, 50937 K\"oln, Germany}}}
\newcommand{\mpia}{\affiliation{\it{Max-Planck-Institute for Astronomy, Koenigstuhl 17, 69117 Heidelberg, Germany}}}
\newcommand{\agnesscott}{\affiliation{\it{Agnes Scott College, 141 E. College Ave., Decatur, GA 30030}}}
\newcommand{\chile}{\affiliation{\it{Departamento de Astronom{\'i}a, Universidad de Chile, Casilla 36-D, Santiago, Chile}}}
\newcommand{\leiden}{\affiliation{\it{Leiden Observatory, Leiden University, PO Box 9513, NL-2300 RA Leiden, the Netherlands }}}
\newcommand{\mpe}{\affiliation{\it{Max-Planck-Institut für extraterrestrische Physik, D-85748 Garching, Germany}}}
\newcommand{\boston}{\affiliation{\it{Boston University Astronomy Department, 725 Commonwealth Avenue, Boston, MA 02215, USA}}}
\newcommand{\cfa}{\affiliation{\it{Harvard-Smithsonian Center for Astrophysics, 60 Garden St. Cambridge, MA 02138}}}
\newcommand{\usf}{\affiliation{\it{University of South Florida, Physics Department, 4202 East Fowler Ave, ISA 2019 Tampa, FL 33620}}}
\newcommand{\uconn}{\affiliation{\it{University of Connecticut, Department of Physics, 2152 Hillside Rd., Storrs, CT 06269}}}

\newcommand{\jao}{\affiliation{\it{Joint ALMA Observatory, Alonso de Córdova 3107, Vitacura, Santiago, Chile}}}
\newcommand{\naoj}{\affiliation{\it{National Astronomical Observatory of Japan, Alonso de Córdova 3788, 61B Vitacura, Santiago, Chile}}}
\newcommand{\naojtwo}{\affiliation{\it{National Astronomical Observatory of Japan, 2-21-1 Osawa, Mitaka,Tokyo, 181-8588, Japan}}}

\author[0000-0001-6431-9633]{Adam Ginsburg}
\nraojansky
\eso

%\author{
%Adam Ginsburg{\nrao},
%\begin{flushleft}
%\institutions
%\end{flushleft}
%        }
%
%\institute{
%    {\nrao}{\it{National Radio Astronomy Observatory, Socorro, NM 87801 USA\\
%                      \email{aginsbur@nrao.edu} 
%                      }} \\
%    }
%
\correspondingauthor{Adam Ginsburg}
\email{aginsbur@nrao.edu; adam.g.ginsburg@gmail.com}


\author{John Bally}
\casa

\author{Ashley Barnes}
\ljmu

\author{Nate Bastian }
\ljmu

\author{Cara Battersby}
\cfa
\uconn

\author{Henrik Beuther}
\mpia

\author{Crystal Brogan}
\nraocv

\author{Yanett Contreras}
\leiden


\author{Joanna Corby}
\nraocv
\usf

\author{Jeremy Darling}
\casa

\author{Chris De~Pree}
\agnesscott

\author{Roberto Galv{\'a}n-Madrid}
\morelia


\author{Guido Garay}
\chile

\author{Jonathan Henshaw}
\mpia


\author{Todd Hunter}
\nraocv

\author{J.~M.~Diederik Kruijssen}
\zah

\author{Steven Longmore}
\ljmu

\author{Xing Lu}
\naojtwo

\author{Fanyi Meng}
\koeln

\author{Elisabeth A.C. Mills }
\sjsu
\boston

\author{Juergen Ott}
\nrao

\author{Jaime E. Pineda}
\mpe



\author{{\'A}lvaro S{\'a}nchez-Monge}
\koeln
\author{Peter Schilke}
\koeln
\author{Anika Schmiedeke}
\koeln
\mpe





\author{Daniel Walker}
\ljmu
\jao
\naoj

\author{David Wilner}
\cfa

\begin{abstract}
We report the detection of $>100$ sources in the Sgr B2 clouds with extents
smaller than 5000 AU.  These sources are most likely to be protostars or
centrally condensed prestellar cores.  The spatial distribution of these sources
demonstrates that Sgr B2 is experiencing a highly extended star formation
event, not just an isolated `starburst' within the protocluster regions M, N,
and S.
\end{abstract}

\ifpdf
\maketitle
\fi



\section{Observation and Data Reduction}
Data were acquired as part of ALMA project 2013.1.00269.S.  Observations were
taken with the 12m Total Power array, the ALMA 7m array, and in two
configurations with the ALMA 12m array.  The setup included the maximum allowed
number of channels, 30720, across 4 spectral windows in a single polarization;
the single-polarization mode was adopted to support moderate spectral resolution
across the broad bandwidth.

The ALMA QA2 calibrated measurement sets were combined to make a single
high-resolution, high-dynamic range data set.  We imaged the continuum jointly
across all four bands, and found that the central regions surrounding Sgr B2M
were severely affected by artifacts that could not be cleaned out.  We
therefore ran 3 iterations of phase-only self-calibration and one iteration of
amplitude + phase self-calibration to yield a substantially improved image.
The total dynamic range, measured as the peak brightness in Sgr B2 to the RMS
noise in a signal-free region of the image, is 22000 (noise $\sim0.08$
mJy/beam), while the dynamic range within one primary beam ($\sim0.5$\arcmin)
of Sgr B2M is only 3700 (noise $\sim0.5$ mJy/beam).  Because of the dynamic
range limitations, and an empirical determination that clean did not converge
if allowed to go too deep, we cleaned to a threshold of 0.5 mJy/beam across the
image.  We performed this same process for both the longest-baseline data only
(resolution $\sim0.5\arcsec$, largest angular scale theoretically 15\arcsec\
[the shortest baseline] but more practically $\sim7$\arcsec\ [the 5th percentile
baseline length]) and the merged 7m + two 12m configuration data.  The merged
data are more useful for studying extended structures but have lower dynamic
range, while the long-baseline-only data are excellent for extracting and
analyzing pointlike or compact sources.

We also produced cubes of all of the spectral lines.  These were lightly
cleaned with only 200 iterations of cleaning.  No self-calibration was applied.
Before continuum subtraction, dynamic range related artifacts similar to those
in the continuum images were present, but these structures are identical across
frequencies, and were therefore removable in the image domain.  We use
median-subtracted cubes for the majority of our analysis, noting that the only
location in which an error on the continuum $>5\%$ is expected is the Sgr B2
North core \citep{Sanchez-Monge2017a}.

\section{Analysis}

\subsection{Continuum Source Identification}
We selected continuum point sources as candidate cores or protostars by eye.
An automated selection is not viable across the majority of the field because
there are many extended \hii regions that dominate the overall map emission.  A
future automated selection algorithm may work if images at comparable
resolution at other frequencies become availabe; the \hii-region sources could
then be excluded.  Additionally, however, there are substantial imaging
artifacts produced by the extremely bright emission sources in Sgr B2 M ($S_{3
mm,max} > 0.8$ Jy) and Sgr B2 N ($S_{3 mm,max} > 0.3$ Jy) that make automated
source identification particularly challenging in the most source-dense regions.

Because the noise varies significantly across the map, a uniform selection
criterion is not possible.  We therefore include two levels of source
identification, `high confidence' sources, which are selected conservatively in
regions of low-background, and `low-confidence' sources that are somewhat lower
signal-to-noise and are often in regions with higher background. 

We measure the local noise for each source by taking the median absolute
deviation in an annulus 0.5 to 1.5 \arcsec around the source center.
All but 7 sources have signal-to-local-noise ratios $S/N>7$.  These
sources are all in regions of particularly high background or source
density and therefore have overestimated local noise.

Our selection criteria result in a reliable but potentially incomplete catalog.

\section{Results}

We detected \ncores compact continuum sources.  Their flux distribution is
shown in Figure \ref{fig:fluxhist}.


\subsection{Source Classification}
For the majority of the detected sources, we have only a continuum detection at
3 mm.  No lines are detected toward most of the sources, especially the faint
ones.  A subset have detections at other bands and can be classified,
especially those associated with \hii regions detected at 0.7 and 1.3 cm
\citep{Gaume1995a,Mehringer1995b,de-Pree1996a,Pree2015a}.  In this section,
we use other means to classify our sample.

% core_flux_distributions
\Figure{figures/core_peak_fluxdensity_powerlawfit.png}
{A histogram of the peak flux density of the observed sources excluding known
\hii regions with a powerlaw fit shown.  The fitted powerlaw is an excellent
fit to the data, but is far shallower than the IMF slope, with
$\alpha=1.94\pm0.07$.  The two brightest regions are Sgr B2M f1 and Sgr B2N K2,
which may be dominated by free-free emission but likely also contain a large
dust mass.}
{fig:fluxhist}{1}{\textwidth}

We first note some key properties of dust at 3 mm.   At 8.4 kpc, a 1 mJy source
corresponds to an optically thin dust mass of $M(40\mathrm{K})=18$ \msun or
$M(20\mathrm{K})=38$ \msun assuming a dust opacity index $\beta=1.75$ to
extrapolate the \citet{Ossenkopf1994a} opacity to $\kappa_{3mm}=0.0018$ cm$^2$
g$^{-1}$.  Our dust-only 5-$\sigma$ sensitivity limit at 40 K therefore ranges
from $M>7$ \msun (0.5 mJy) to $M>45$ \msun (2.5 mJy) across the map.  If we
were to assume that these are all cold, dusty sources, as is typically (and
reasonably) assumed for local clouds, they would be extremely massive and
dense, with the lowest measurable density being $n(40\mathrm{K}) > 3\ee{6}$
\percc (corresponding to 7 \msun in a 0.5\arcsec radius sphere).  Such extreme
objects are possible, but since we have detected $>100$ of these sources, it
makes sense to evaluate other possibilities.


\subsubsection{Alternative 1: The sources are externally ionized gas blobs}
One possibility is that these sources are not dusty at all, nor pre- or
protostellar, but are instead the brightest compact clumps surrounding \hii
regions.  They would then be analogous to the heads of cometary clouds,
externally ionized globules (``EGGs"), or proplyds, and their observed emission
would give no clue to their nature because the light source is extrinsic.

The majority of the detected sources have size $<4000$ AU, i.e., they are
unresolved.  By contrast, the free-floating EGGs so far observed have sizes
10,000-20,000 AU \citep{Sahai2012a,Sahai2012b}, so they would be resolved in
our observations.  Toward the brightest frEGG in Cygnus X, \citet{Sahai2012b}
measured a peak intensity $S_{8.5 GHz} \approx 1.5$ mJy/beam in a
$\approx3\arcsec$ beam.  Cygnus X is $6\times$ closer that the Galactic center,
so their beam size is the same physical scale as ours.  If the free-free
emission is thin, the brightness in our data would be $S_{95 GHz} =
(95/8.5)^{-0.1} S_{8.5 GHz} = 0.79 S_{8.5 GHz} \approx 1.2$ mJy/beam.  These
frEGGs would be detectable in our data.  Comparison to radio observations
at a comparable resolution will be needed to rule out the externally ionized
globule hypothesis for resolved regions.

If the detected sources were either EGGs or cometary clouds, they should be
located within or near \hii regions.  Many of the sources are near \hii
regions, as seen in Figure \ref{fig:coreson20cm}.  However, they are also
nearly all associated with a ridge of HC$_3$N emission (Figure
\ref{fig:coresonhc3n}).  If they are deeply embedded within the molecular
material, they cannot be externally ionized.  The current data do not provide
enough information on the geometry of the clouds to rule out the possibility
that the point sources are just illuminated cloud edges, but the fact that the
ionized gas is brightest adjacent to, rather than on top of, the HC$_3$N
suggests that the HC$_3$N traces a full molecular cloud rather than a thin
PDR-like layer.

%This
%scale is consistent with that of frEGGs seen in Carina \citep[e.g.][]{Sahai2012a}.
%The flux density of our sources is 1-2 orders of magnitude below what we would
%predict from the dust observed in the \citet{Sahai2012a} object, though the
%free-free... \todo{Look up observed free-free emission from proplyds and frEGGs.
%Could we have frEGGs?}

% The `tadpole' \citep{Sahai2012b} has $S_{22 \mathrm{GHz}} = 30$ mJy at d=1.4 kpc, resolved
% to $\sim10\times10$ arcsec.  In the CMZ... this would be visible.  Maybe compare to Zadeh's
% recent obs of "proplyds"?

% The key requirement, if these are all indeed externally ionized molecular
% blobs, is that of a strong ionizing radiation field.  Indeed, L-band
% observations \citep{Yusef-Zadeh2004a} show that many of these sources are
% located along the outer edges of a large-scale \hii region (Figure
% \ref{fig:coreson20cm}).
% 
% % However, their location poses a problem for the
% % frEGG/proplyd hypothesis: % this next sentence doesn't really make sense...
% %these sources are observed \emph{within} \hii
% %regions, not along their outskirts. 
% However, when analogous structures are present
% along the outskirts of clouds, they are usually accompanied by a larger
% contiguous cloud edge, which results in a continuous sharp-edged bright feature
% corresponding to a PDR.  Such edges are seen in the Orion Bar and M16's
% ``Pillars of Creation''.   We do not observe any such features here.
% 
% We know from our molecular line observations in HC$_3$N that most of the
% continuum sources lay along a molecular ridge (TODO: FIGURE), so it appears
% most of these sources are embedded in molecular material.

TODO: H41a TE peak map.  Show locations relative to confirmed HII regions

% pointsource_overlay
\FigureOneCol{figures/cores_on_20cm_continuum.png}
{The location of the detected continuum sources (red points) overlaid on a 20
cm continuum VLA map highlighting the diffuse free-free (or possibly
synchrotron) emission in the region \citep{Yusef-Zadeh2004a}.}
{fig:coreson20cm}{1}{0.5\textwidth}

% pointsource_overlay
\FigureOneCol{figures/cores_on_hc3n_peak.png}
{The location of the detected continuum sources (red points) overlaid on a map
of the HC$_3$N peak intensity.  HC$_3$N traces moderate-density molecular gas.}
{fig:coresonhc3n}{1}{0.5\textwidth}

\subsubsection{Alternative 2: The sources are \hii regions produced by interloper ionizing stars}
If there is a large population of older, but still ionizing, stars, they could
ignite \hii regions when they fly through molecular material.  See
\ref{sec:theyarehiiregions} for calculations of stationary \hii region
properties.  The main problem with this scenario is the spatial distribution of
the observed sources.  While most of them are associated with dense gas and dust ridges,
not all of the high-column molecular gas regions have such sources in them.  If
there is a free-floating population of OB stars responsible for the 3 mm point
source population, their distribution should match that of the gas.  Also,
there is no such population of sources seen outside of the dense gas in the
infrared, which again we should expect if there is a uniformly distributed
population.  

\subsubsection{Alternative 3: The sources are \hii regions produced by recently-formed OB stars}
\label{sec:theyarehiiregions}

For an unresolved spherically symmetric \hii region ($R=4000$ AU), the expected flux density
is $S_{95 \mathrm{GHz}} = 4.7$ mJy for a $Q_{lyc}=10^{47}$ \pers source
(assuming $T_e=7000$ K), and that value scales linearly with $Q_{lyc}$ as long
as the source is optically thin.  An extremely compact \hii region, e.g., one
with $R<100$ AU and corresponding density $n>10^6$ \percc, would be optically
thick and therefore fainter, $S_{95 \mathrm{GHz}}(R=100 \mathrm{AU},
Q_{lyc}=10^{47} \pers)=3.4$ mJy.  Even the brightest O-stars could produce \hii
regions as faint as 0.5 mJy if embedded in extremely high density gas; above
$Q_{lyc}>10^{47}$ \pers, a 25 AU \hii region would be $\sim0.5$ mJy.

Figure \ref{fig:hiibrightness} shows the predicted brightness for various \hii
regions produced by OB stars and the density of to those \hii regions.  In
order for the detected sources to be O-star-driven \hii regions, with $10^{47}
< Q_{lyc} < 10^{50}$ \pers, they must be optically thick and therefore
extremely compact and dense.  There is a narrow range of late O/early B stars,
$10^{46} < Q_{lyc} < 10^{47}$ \pers, that could be embedded in compact \hii
regions of almost any size and produce the observed range of flux densities.
Anything fainter, later than $\sim$B0 ($Q_{lyc}<10^{46}$ \pers), would be
incapable of producing the observed flux densities.

This restrictive parameter space, combined with a steep luminosity
function that implies there are many more sources at slightly lower luminosity,
is evidence against the population being dominated by \hii regions.

\Figure{figures/HII_region_brightness.png}
{Simple models of spherical \hii regions to illustrate the observable
properties of such regions.  The \hii region size is shown by line color; the legend
in the left plot applies to both figures.  (left) The expected brightness temperature (left
axis) and corresponding flux density within a FWHM=0.5\arcsec beam (right axis) as a
function of the Lyman continuum luminosity for a variety of source radii.
(right) The density required to produce an \hii region of that radius.  The
horizontal dashed line shows the density corresponding to an unresolved dust
source at the 5-$\sigma$ detection limit ($\approx0.5$ mJy, or about 10 \msun
of dust,
assuming $T=40$ K).  Above this line, dust emission would dominate over
free-free emission.  The dotted line shows the density required for dust
emission to produce a 10 mJy source at $T=40$ K.  
As seen in the left plot, for any moderate-sized \hii region, $R>100$ AU, a high-luminosity
star ($Q_{lyc} > 10^{47}$ \pers) would produce an \hii region brighter than the
majority of our sample, which includes only a few sources brighter than 10 mJy.
The densities required to produce \hii regions within our observed range
($1<S_\nu<10$ mJy) are fairly extreme, $n\gtrsim10^6$ \percc, for O-stars.}
{fig:hiibrightness}{1}{18cm}


\subsubsection{The sources are protostars}
After ruling out the other hypotheses, we conclude that these sources are
predominantly embedded protostars.  Their emission is likely dust-dominated,
but is probably warmer than the cloud average $20\sim40$ K.

Some validation of the protostellar hypothesis comes from catalog matching.
The \citet{Caswell2010a} Methanol Multibeam Survey identified 11 sources in our
observed field of view, of which 10 have a clear match in our catalog.
Others of the sources have been identified with known \hii regions from
\citet{Gaume1995a}.


We compare our detected sample to that of the Herschel Orion Protostar Survey
\citep[HOPS;][]{Furlan2016a} in order to get a general sense of what types of
sources we have detected.  Figure \ref{fig:hopshist} shows the HOPS source
fluxes at 870\um scaled to 3 mm assuming a dust opacity index $\beta=1.5$,
which is shallower than usually inferred.  The 870\um data were acquired with a
$\sim20\arcsec$ FWHM beam, which translates to a resolution $\sim1\arcsec$ at
8.4 kpc, so our beam size is similar to theirs.  The HOPS sources are all
fainter than the Sgr B2 sources.  The brightest HOPS source would only be 0.2
mJy in Sgr B2, or about a 4-$\sigma$ source - below our detection threshold
even in the noise-free regions of the map.  We can therefore conclude that
the Sgr B2 sources are much more luminous and therefore massive protostars.

% Bontemps+ 2010: 3.5mm flux of N63 ~ 36 mJy, -> 1 mJy @ Sgr B2

\Figure{figures/core_peak_intensity_histogram_withHOPS.png}
{A histogram combining the detected Sgr B2 cores with predicted flux densities
based on the HOPS \citep{Furlan2016a} survey.  The HOPS histogram shows the 870
\um data from that survey scaled to 3 mm assuming $\beta=1.5$.  Every HOPS
source is well below the detection threshold for our observations.}
{fig:hopshist}{1}{15cm}

The flux density distribution of the non-\hii region sources follows a powerlaw
with slope $\alpha=1.94\pm0.07$ \citep[fitted with the MLE method
of][]{Clauset2007a}.  If we assume that the stellar mass is linearly
proportional to the 3 mm continuum flux density, this measurement implies a
slope shallower than the $\alpha\sim2.35$ expected for a normal IMF.  
It is possible that the IMF is genuinely different here, but it is 
more likely that the more massive stars are surrounded by warmer gas.

If we make the very simplistic assumptions that the sources we detect are all
$L\gtrsim2000$ \lsun ($M\gtrsim8 \msun$), 
we can infer the total (proto)stellar mass in the observed region.
Using a \citet{Kroupa2001a} mass function with $M_{max}=200$ \msun, 23\% of
the mass is contained in $M>8\msun$ stars.  Using $M=8\msun$ as the lower-limit
case for each source, the identified sources have total mass $M(>8)=1800\msun$.
The total stellar mass implied is $M_{tot} = 8\ee{3}$ \msun.  If we use the
mean stellar mass for $M>8$ \msun, $\bar{M}=21.1$ \msun, then $M_{tot}=2\ee{4}$
\msun.





% While we identified \ncores sources from the continuum data, since we have only
% a single continuum band available, it is difficult to classify most of these
% except to say that they are certainly forming or recently formed stars.
% However, for a small subset, we have spectral line detections in either
% molecular or ionized species that tell us qualitatively whether a source is
% ionizing an \hii region or is surrounded by interesting molecular species.
% 
% \todo{Continue here - give the subset of sources with good line IDs (which is
% probably a by-hand process) and show example spectra of something not Sgr B2 M
% or N}

\subsection{An examination of star formation thresholds}
\subsubsection{Comparison to other CMZ clouds}
Lada et al, and others, have proposed that star formation can only occur above
a certain density or column density threshold\footnote{Column density is more
commonly used because of its observational convenience, but it is physically
meaningless unless high column density leads to high optical depths and thereby
changes the gas's ability to cool.}.  In G0.253+0.016, very little star formation
has been observed \citep{Longmore2013a,Johnston2014a,Rathborne2015a} despite
most of the cloud existing above the locally measured column density threshold.

Since we have detected substantial ongoing star formation in the form of
high-mass protostars and/or protostellar cores, we can assess where these stars
form and whether the same (lack of) a threshold exists in Sgr B2.  We therefore
plot the column density distribution (flux distribution?) and overlay the
cumulative distribution function of the background brightness around the cores.

Comparing Sgr B2 to G0.253, the majority of the Sgr B2 cloud is brighter and at
higher column than G0.253.  The presence of star formation in Sgr B2 nearly all
occurs at a higher column than exists within G0.253 (Figure
\ref{fig:fluxhist}).  The lack of SF in the brick is therefore consistent with
the active SF in Sgr B2 and the CMZ's higher SF threshold is confirmed.

\subsubsection{Comparison to Lada, Lombardi, and Alves 2010}
In this section, we compare the star formation threshold in Sgr B2 to that in
local clouds performed by \citet{Lada2010a}.  They determined that all star
formation in local clouds occurs above a column density threshold $M_{thresh} >
116$ \msun pc$^{-2}$, or $N_{thresh}(\hh) > 5.2\ee{21}$ \persc assuming the
mean particle mass is 2.8 amu \citep{Kauffmann2008a}.  We first note, then,
that \emph{all pixels} in our column density maps are above this threshold
by \emph{at least} a factor of 10.

However, the CMZ is 8.4 kpc away from us in the direction of our Galaxy's
center, meaning there is a potentially enormous amount of material unassociated
with the Sgr B2 cloud along the line of sight.  This material may be as low as
5\ee{21} \persc or as high as 5\ee{22} \persc.  The former value corresponds to
the background at high latitudes, $b\sim0.5$, while the latter  is
approximately the lowest seen within our field of view.  If the higher value is the 
correct foreground, there must be a perfect vacuum surrounding the dense gas in
the Sgr B2 cloud - the `hole' seen in Fig...TODO: show a hole figure.
Even with the very aggressive foreground value of 5\ee{22} \persc subtracted,
essentially the whole cloud exists above this threshold.  We can therefore
immediately rule out the possibility that there is a universal star formation
column threshold, since a large fraction of the observed volume exhibits
no hint at all of star formation activity.

-What kind of stars are we sensitive to?  Are they?
(this is now handled above)


\Figure{figures/flux_histograms_with_core_location_CDF.png}
{Histograms of the brightness measured with a variety of instruments at
different submillimeter bands with the cumulative distribution function (CDF)
of the \emph{background} brightness surrounding each core superposed.  The
X-axis units are arbitrary (because right now I don't know the units of all of
these) except for column, which is in units of cm$^{-2}$ of \hh as derived from
SED fits to Herschel data (Battersby+).  The grey line is of the observed
region in Sgr B2 and the blue line is of G0.253+0.016.  The thick grey line is
the CDF of core background brightness, and is labeled by the right axis.}
{fig:fluxhist}{1}{\textwidth}



%\ifstandalone
%\bibliographystyle{apj_w_etal}  % or "siam", or "alpha", or "abbrv"
\bibliographystyle{aasjournal}  % or "siam", or "alpha", or "abbrv"
%\bibliography{thesis}      % bib database file refs.bib
%\bibliography{bibdesk}      % bib database file refs.bib
\bibliography{extracted}      % bib database file refs.bib
%\fi


\end{document}

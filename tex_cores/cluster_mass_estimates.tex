\begin{table*}[htp]
\centering
\caption{Cluster Masses}
\begin{tabular}{cccccccccc}
\label{tab:clustermassestimates}
Name & $N(cores)$ & $N(H\textsc{ii})$ & $M_{count}$ & $M_{inferred}$ & $M_{inferred, H\textsc{ii}}$ & $M_{inferred, cores}$ & $M_{count}^s$ & $M_{inf}^s$ & SFR \\
 &  &  & $\mathrm{M_{\odot}}$ & $\mathrm{M_{\odot}}$ & $\mathrm{M_{\odot}}$ & $\mathrm{M_{\odot}}$ & $\mathrm{M_{\odot}}$ & $\mathrm{M_{\odot}}$ & $\mathrm{M_{\odot}\,kyr^{-1}}$ \\
\hline
M & 20 & 51 & 2600 & 9700 & 17000 & 2700 & 1295 & 20700 & 13 \\
N & 12 & 7 & 460 & 2000 & 2300 & 1600 & 150 & 2400 & 2.6 \\
NE & 4 & 2 & 140 & 600 & 650 & 540 & 52 & 1200 & 0.81 \\
S & 5 & 1 & 110 & 500 & 330 & 680 & 50 & 1100 & 0.68 \\
Total & 235 & 72 & 6100 & 28000 & 24000 & 32000 & 1993 & 33400 & 37 \\
\hline
\end{tabular}
\par
$M_{count}$ is the mass of directly counted protostars, assuming each millimeter source is 12.0 \msun, or 45.5 \msun if it is also an \hii region.  $M_{inferred,cores}$ and $M_{inferred,\hii}$ are the inferred total stellar masses assuming the counted objects represent fractions of the total mass 0.09 (cores) and 0.14 (\hii regions).  $M_{inferred}$ is the average of these two.  $M_{count}^s$ and $M_{inf}^s$ are the counted and inferred masses reported in \citet{Schmiedeke2016a}.  The star formation rate is computed using an age $t=0.74$ Myr, which is the time of the last pericenter passage in the \citet{Kruijssen2015a} model.  The \emph{total} column represents the total over the whole observed region.    The clusters sum to much less than the \emph{total} because the Deep South region is not included, and it dominates the overall core count.
\end{table*}

\begin{table*}[htp]
\centering
\begin{minipage}{130mm}
\caption{Cluster Masses}
\begin{tabular}{ccccccccccc}
\label{tab:clustermassestimates}
Name & $N({\rm cores})$ & $N({\rm H\textsc{ii}})$ & $M_{\rm count}$ & $M_{\rm inferred}$ & $M_{\rm inferred,max}$ & $M_{\rm inferred, H\textsc{ii}}$ & $M_{\rm inferred, cores}$ & $M_{\rm count}^s$ & $M_{\rm inf}^s$ & SFR \\
 &  &  & $\mathrm{M_{\odot}}$ & $\mathrm{M_{\odot}}$ & $\mathrm{M_{\odot}}$ & $\mathrm{M_{\odot}}$ & $\mathrm{M_{\odot}}$ & $\mathrm{M_{\odot}}$ & $\mathrm{M_{\odot}}$ & $\mathrm{M_{\odot}\,yr^{-1}}$ \\
\hline
M & 17 & 47 & 2300 & 8800 & 15000 & 15000 & 2300 & 1295 & 20700 & 0.012 \\
N & 11 & 3 & 270 & 1200 & 1500 & 980 & 1500 & 150 & 2400 & 0.0017 \\
NE & 4 & 0 & 48 & 270 & 540 & 0 & 540 & 52 & 1200 & 0.00037 \\
S & 5 & 1 & 110 & 500 & 680 & 330 & 680 & 50 & 1100 & 0.00068 \\
Unassociated & 203 & 6 & 0 & 15000 & 27000 & 2000 & 27000 & - & - & 0.02 \\
Total & 0 & 0 & 0 & 0 & 46000 & 0 & 0 & 1993 & 33400 & 0 \\
\hline
\end{tabular}
\par
$M_{\rm count}$ is the mass of directly counted protostars, assuming each millimeter source is 12.0 \msun, or 45.5 \msun if it is also an \hii region.  $M_{\rm inferred,cores}$ and $M_{\rm inferred,\hii}$ are the inferred total stellar masses assuming the counted objects represent fractions of the total mass 0.09 (cores) and 0.14 (\hii regions).  $M_{\rm inferred}$ is the average of these two.  $M_{\rm count}^s$ and $M_{\rm inf}^s$ are the counted and inferred masses reported in \citet{Schmiedeke2016a}.  The star formation rate is computed using $M_{\rm inferred}$ and an age $t=0.74$ Myr, which is the time of the last pericenter passage in the \citet{Kruijssen2015a} model.  The \emph{Total} column represents the total over the whole observed region.  The \emph{Total}$_{\rm max}$ column takes the higher of $M_{\rm inferred,\hii}$ and $M_{\rm inferred,cores}$ from each row and sums them.  We have included \hii regions in the $N(\hii)$ counts  that are \emph{not} included in our source table \ref{tab:photometry} because they are too diffuse, or because they are unresolved in our data but were resolved in the \citet{De-Pree2014a} VLA data.  As a result, the total source count is greater than the source count reported in Table \ref{tab:photometry}. Also, the unassociated \hii region count is incomplete; it is missing both diffuse \hii regions and possibly unresolved hypercompact \hii regions, since there are no VLA observations comparable to \citet{De-Pree2014a} in the unassociated regions.
\end{table*}

\begin{table*}[htp]
\centering
\caption{Cluster Masses}
\begin{tabular}{cccccccccc}
\label{tab:clustermassestimates}
Name & $N(cores)$ & $N(H\textsc{ii})$ & $M_{obs}$ & $M_{inferred}$ & $M_{inferred, H\textsc{ii}}$ & $M_{inferred, cores}$ & $M_{obs}^s$ & $M_{inf}^s$ & SFR \\
 &  &  & $\mathrm{M_{\odot}}$ & $\mathrm{M_{\odot}}$ & $\mathrm{M_{\odot}}$ & $\mathrm{M_{\odot}}$ & $\mathrm{M_{\odot}}$ & $\mathrm{M_{\odot}}$ & $\mathrm{M_{\odot}\,kyr^{-1}}$ \\
\hline
M & 18 & 51 & 2500 & 9500 & 17000 & 2400 & 1295 & 20700 & 13 \\
N & 10 & 7 & 440 & 1800 & 2300 & 1400 & 150 & 2400 & 2.5 \\
NE & 4 & 2 & 140 & 600 & 650 & 540 & 52 & 1200 & 0.81 \\
S & 3 & 1 & 81 & 370 & 330 & 410 & 50 & 1100 & 0.5 \\
Total & 221 & 72 & 5900 & 27000 & 24000 & 30000 & 1993 & 33400 & 36 \\
\hline
\end{tabular}
\par
$M_{obs}$ is the mass of directly observed protostars, assuming each millimeter source is 12.0 \msun, or 45.5 \msun if it is also an \hii region.  $M_{inferred,cores}$ and $M_{inferred,\hii}$ are the inferred total stellar masses assuming the counted objects represent fractions of the total mass 0.09 (cores) and 0.14 (\hii regions).  $M_{inferred}$ is the average of these two.$M_{obs}^s$ and $M_{inf}^s$ are the observed and inferred masses reported in \citet{Schmiedeke2016a}.
\end{table*}
